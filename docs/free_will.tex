% Options for packages loaded elsewhere
\PassOptionsToPackage{unicode}{hyperref}
\PassOptionsToPackage{hyphens}{url}
\PassOptionsToPackage{dvipsnames,svgnames,x11names}{xcolor}
%
\documentclass[
  a4paper,
]{report}
\usepackage{amsmath,amssymb}
\usepackage{lmodern}
\usepackage{iftex}
\ifPDFTeX
  \usepackage[T1]{fontenc}
  \usepackage[utf8]{inputenc}
  \usepackage{textcomp} % provide euro and other symbols
\else % if luatex or xetex
  \usepackage{unicode-math}
  \defaultfontfeatures{Scale=MatchLowercase}
  \defaultfontfeatures[\rmfamily]{Ligatures=TeX,Scale=1}
\fi
% Use upquote if available, for straight quotes in verbatim environments
\IfFileExists{upquote.sty}{\usepackage{upquote}}{}
\IfFileExists{microtype.sty}{% use microtype if available
  \usepackage[]{microtype}
  \UseMicrotypeSet[protrusion]{basicmath} % disable protrusion for tt fonts
}{}
\makeatletter
\@ifundefined{KOMAClassName}{% if non-KOMA class
  \IfFileExists{parskip.sty}{%
    \usepackage{parskip}
  }{% else
    \setlength{\parindent}{0pt}
    \setlength{\parskip}{6pt plus 2pt minus 1pt}}
}{% if KOMA class
  \KOMAoptions{parskip=half}}
\makeatother
\usepackage{xcolor}
\usepackage[top=20mm,left=30mm,heightrounded]{geometry}
\usepackage{longtable,booktabs,array}
\usepackage{calc} % for calculating minipage widths
% Correct order of tables after \paragraph or \subparagraph
\usepackage{etoolbox}
\makeatletter
\patchcmd\longtable{\par}{\if@noskipsec\mbox{}\fi\par}{}{}
\makeatother
% Allow footnotes in longtable head/foot
\IfFileExists{footnotehyper.sty}{\usepackage{footnotehyper}}{\usepackage{footnote}}
\makesavenoteenv{longtable}
\usepackage{graphicx}
\makeatletter
\def\maxwidth{\ifdim\Gin@nat@width>\linewidth\linewidth\else\Gin@nat@width\fi}
\def\maxheight{\ifdim\Gin@nat@height>\textheight\textheight\else\Gin@nat@height\fi}
\makeatother
% Scale images if necessary, so that they will not overflow the page
% margins by default, and it is still possible to overwrite the defaults
% using explicit options in \includegraphics[width, height, ...]{}
\setkeys{Gin}{width=\maxwidth,height=\maxheight,keepaspectratio}
% Set default figure placement to htbp
\makeatletter
\def\fps@figure{htbp}
\makeatother
\setlength{\emergencystretch}{3em} % prevent overfull lines
\providecommand{\tightlist}{%
  \setlength{\itemsep}{0pt}\setlength{\parskip}{0pt}}
\setcounter{secnumdepth}{5}
\ifLuaTeX
\usepackage[bidi=basic]{babel}
\else
\usepackage[bidi=default]{babel}
\fi
\babelprovide[main,import]{ngerman}
% get rid of language-specific shorthands (see #6817):
\let\LanguageShortHands\languageshorthands
\def\languageshorthands#1{}
\ifLuaTeX
  \usepackage{selnolig}  % disable illegal ligatures
\fi
\usepackage[]{natbib}
\bibliographystyle{plainnat}
\IfFileExists{bookmark.sty}{\usepackage{bookmark}}{\usepackage{hyperref}}
\IfFileExists{xurl.sty}{\usepackage{xurl}}{} % add URL line breaks if available
\urlstyle{same} % disable monospaced font for URLs
\hypersetup{
  pdftitle={Freier Wille als Illusion oder Notwendigkeit},
  pdfauthor={Emma Erhard, Universität Konstanz; Hamilkar Constantin Oueslati, Universität Konstanz; Naima Steimel, Universität Konstanz},
  pdflang={de},
  colorlinks=true,
  linkcolor={Maroon},
  filecolor={Maroon},
  citecolor={Blue},
  urlcolor={Blue},
  pdfcreator={LaTeX via pandoc}}

\title{Freier Wille als Illusion oder Notwendigkeit}
\author{Emma Erhard, Universität Konstanz \and Hamilkar Constantin Oueslati, Universität Konstanz \and Naima Steimel, Universität Konstanz}
\date{14.01.2023}

\begin{document}
\maketitle

\renewcommand*\contentsname{Inhaltsverzeichnis}
{
\hypersetup{linkcolor=}
\setcounter{tocdepth}{3}
\tableofcontents
}
\hypertarget{preface}{%
\chapter*{Vorwort}\label{preface}}
\addcontentsline{toc}{chapter}{Vorwort}

\hypertarget{about}{%
\section*{Über dieses Webbook}\label{about}}
\addcontentsline{toc}{section}{Über dieses Webbook}

Dieses Webbook wurde im Rahmen des Seminars ``Von Freiheit und Notwendigkeit'' (WS 2022/2023 - Universität Konstanz) erstellt. Es enthält alle Informationen und Materialien, welche die Teilnehmer*innen des Seminars zur Mitarbeit in der von uns gestalteten Sitzung mit dem Ttel ``Freier Wille als Illusion oder Notwendigkeit'' benötigen.

Die wichtigsten Ergebnisse etwaiger Diskussionen im Rahmen besagter Sitzung werden ebenfalls in diesem Webbook dokumentiert.

Bei Fragen zu den Inhalten dieses Webboks bzw. der Sitzung zögern Sie bitte nicht die Autor*innen zu kontaktieren.

\hypertarget{authors}{%
\section*{Die Autor*innen}\label{authors}}
\addcontentsline{toc}{section}{Die Autor*innen}

\hypertarget{eerhard}{%
\subsection*{Emma Erhard}\label{eerhard}}
\addcontentsline{toc}{subsection}{Emma Erhard}

Pronomen: sie/ihr (dt.) bzw. she/her (engl.)\\
Studierender Mensch - Universität Konstanz\\
Studienfach: Psychologie (M.Sc.)\\

Mail: \href{mailto:emma.erhard@uni-konstanz.de?subject=Freier\%20Wille\%20als\%20Illusion\%20oder\%20Notwendigkeit}{emma.erhard@uni-konstanz.de}

\hypertarget{hcoueslati}{%
\subsection*{Hamilkar Constantin Oueslati}\label{hcoueslati}}
\addcontentsline{toc}{subsection}{Hamilkar Constantin Oueslati}

Pronomen: dey/deren/denen (dt.) bzw. they/their/them (engl.)\\
Studierender Mensch - Universität Konstanz\\
Studienfach: Psychologie (M.Sc.)\\

Mail: \href{mailto:hamilkar-constantin.oueslati@uni-konstanz.de?subject=Freier\%20Wille\%20als\%20Illusion\%20oder\%20Notwendigkeit}{hamilkar-constantin.oueslati@uni-konstanz.de}

Web: \url{https://hco-consulting.eu}

\hypertarget{nsteimel}{%
\subsection*{Naima Steimel}\label{nsteimel}}
\addcontentsline{toc}{subsection}{Naima Steimel}

Pronomen: sie/ihr (dt.) bzw. she/her (engl.)\\
Studierender Mensch - Universität Konstanz\\
Studienfach: Psychologie (M.Sc.)\\

Mail: \href{mailto:naima.steimel@uni-konstanz.de?subject=Freier\%20Wille\%20als\%20Illusion\%20oder\%20Notwendigkeit}{naima.steimel@uni-konstanz.de}

\hypertarget{lecturer}{%
\section*{Der Dozent des Seminars}\label{lecturer}}
\addcontentsline{toc}{section}{Der Dozent des Seminars}

\hypertarget{dbeis}{%
\subsection*{Dr.~Daniel Beis}\label{dbeis}}
\addcontentsline{toc}{subsection}{Dr.~Daniel Beis}

Pronomen: er/ihm (dt.) bzw. he/him (engl.)\\
Wissenschaftlicher Mitarbeiter am Institut für Ernährungswissenschaften\\
Justus-Liebig-Universität Gießen\\

Mail: \href{mailto:?subject=Seminar\%20Freier\%20Wille\%20als\%20Illusion\%20oder\%20Notwendigkeit\%20-\%20Universität\%20Konstanz}{daniel.beis@ernaehrung.uni-giessen.de}

\hypertarget{intro}{%
\chapter{Einführung}\label{intro}}

\hypertarget{freewill-ynmi}{%
\section{Freier Wille? - Ja, Nein, Vielleicht, Egal}\label{freewill-ynmi}}

Im Rahmen dieser Seminarsitzung möchten wir uns mit euch auf kreative Art und Weise mit folgende Fragestellungen auseinandersetzen:

\begin{quote}
\emph{(1) Existiert der freie Wille wirklich oder ist er schlicht eine Illusion?}
\end{quote}

\begin{quote}
\emph{(2) Kann unsere Gesellschaft nur dann funktionieren, wenn der freie Wille existiert?}
\end{quote}

\begin{quote}
\emph{(3) Falls der freie Wille nicht existieren sollte, müssen wir an dessen Existenz glauben, um das Funktionieren unserer Gesellschaft sicherstellen zu können?}
\end{quote}

Die kritische Betrachtung solch abstrakter Fragestellungen ist leider oft alles andere als leicht. Aus diesem Grund wollen wir mit euch die besagten Fragestellungen anhand eines deutlich greifbareren Konzeptes kritisch diskutieren: der \textbf{Schuldfähigkeit}.

\hypertarget{freewill-blame}{%
\section{Freier Wille und Schuldfähigkeit}\label{freewill-blame}}

Bevor wir uns an der Beantwortung der obigen Fragen versuchen, lasst uns zuerst eine Antwort auf \textbf{die Frage} finden, \textbf{die immer die erste sein sollte}:

\begin{quote}
\emph{Wieso ist die Antwort auf diese Frage(n) von Relevanz?}
\end{quote}

Zur Beantwortung der \textbf{ersten Frage} lasst uns einige \textbf{potentielle} Implikationen von zwei Antwortmöglichkeiten für Frage (1) im Hinblick auf die Schuldfähigkeit einer kriminellen Person betrachten.

\textbf{Der freie Wille existiert.}

\begin{itemize}
\item
  Personen sind generell dazu fähig, sich bewusst dafür oder dagegen zu entscheiden ein bestimmtes Verhalten zu zeigen.

  \begin{itemize}
  \tightlist
  \item
    Personen sind im rechtlichen Sinne generell als \textbf{schuldfähig} zu betrachten.
  \end{itemize}
\item
  Personen sind \textbf{verantwortlich} für das Verhalten, welches sie bewusst zeigen bzw. nicht zeigen.

  \begin{itemize}
  \tightlist
  \item
    Personen \textbf{tragen} die \textbf{Schuld} für ihre kriminellen Handlungen bzw. für das kriminelle Unterlassen bestimmter Handlungen.
  \end{itemize}
\end{itemize}

\textbf{Der freie Wille existiert nicht.}

\begin{itemize}
\item
  Ob Personen ein bestimmtes Verhalten zeigen oder nicht, hängt \textbf{nicht} von bewussten Entscheidungsprozessen ab.

  \begin{itemize}
  \tightlist
  \item
    Personen sind im rechtlichen Sinne generell als \textbf{nicht schuldfähi}g zu betrachten.
  \end{itemize}
\item
  Personen sind \textbf{nicht verantwortlich} für das Verhalten, welches sie zeigen bzw. nicht zeigen.

  \begin{itemize}
  \tightlist
  \item
    Personen \textbf{tragen keine Schuld} für ihre kriminellen Handlungen bzw. für das kriminelle Unterlassen bestimmter Handlungen.
  \end{itemize}
\end{itemize}

Eine mögliche Antwort auf die \textbf{erste Frage} lautet dementsprechend:

\begin{quote}
\emph{Die Antworten auf Fragen (1) bis (3) sind von Relevanz, da diese beispielsweise bedeutende Implikationen für die Rechtsprechung in unserer Gesellschaft haben können.}
\end{quote}

\hypertarget{freewill-atcourt}{%
\chapter{Der Freie Wille Vor Gericht}\label{freewill-atcourt}}

Die zentralen Fragestellungen wurden definiert. Wir haben uns für ein greifbares und relevantes Konzept entschieden, anhand dessen wir diese diskutieren möchten. Wir sind uns im Klaren über die Relevanz der besagter Fragestellungen.

Wie können wir nun die genannten Fragestellungen anhand des Konzepts der Schuldfähigkeit auf kreative Art und Weise kritisch diskutieren?

Die Antwort lautet. \emph{Mit einem Rollenspiel!}

\begin{quote}
\textbf{\emph{Die Richter*innen rufen die Anwesenden zur Ordnung. Die Verhandlung beginnt!}}
\end{quote}

\hypertarget{deathofanicon}{%
\section{Die Ermordung einer Ikone}\label{deathofanicon}}

Im Rahmen einer \textbf{fiktiven Gerichtsverhandlung} soll entschieden werden, ob der \textbf{Angeklagte Daniel B.} (unser Dozent) des \textbf{Mordes} an seinem Kollegen \textbf{Sigmund F.} schuldig ist.

\textbf{Die Beweislage scheint eindeutig.} Alle vorhandenen Beweisstücke, die gerichtsmedizinische Untersuchung sowie die überlieferte Zeugenaussage von Carl Gustav J. lassen auf nur einen möglichen Tathergang schließen:

\begin{quote}
\emph{Am Morgen des 16.10.2022 nutzte Daniel B. einen Vortex-Manipulator, um zurück in das Jahr 1907 zu reisen. Am späten Abend des 03.11.1907 verschaffte sich Daniel B. sodann Zugang zu der Wohnung des Opfers in der Berggasse 19 in Wien. Laut der Aussage des Zeugen Carl Gustav J. ``stürmte'' Daniel B. in das Studierzimmer des Opfers und unterbrach unter Verwendung ``äußerst grotesker Flüche'' das erste Treffen zwischen dem Opfer und dem Zeugen. Als sich das Opfer auch nach mehrmaligen Aufforderungen von Daniel B. weigerte seine ``irrsinnige'' Sexualtheorie aufzugeben, begann Daniel B. ``wie besessen'' mit einem schweizer Taschenmesser auf das Opfer Sigmund F. einzustechen. Wenig später erlag Sigmund F. seinen Verletzungen.}
\end{quote}

\hypertarget{blameless}{%
\section{Ein Schuldloser Mord?}\label{blameless}}

Ausgehend von den dem Gericht vorliegenden Beweismitteln und der Zeugenaussage von Car Gustav J. scheint es außer Frage zu stehen, dass Daniel B. Sigmund F. ermordet hat.

Dies bedeutet jedoch nicht, dass Daniel B. des Mordes an Sigmund F. auch schuldig gesprochen werden kann. Lediglich wenn das Gericht basierend auf den Ausführungen der Staatsanwaltschaft und der Verteidigung zu dem Schluss kommen, dass Daniel B. zum Zeitpunkt des Mordes schuldfähig war, kann ein Schuldspruch erfolgen.

Daher bitten die ehrenwerten Richter*innen E. Erhard, H. C. Oueslati und N. Steimel die Teilnehmer*innen dieses Seminars sich in zwei Gruppen (Staatsanwaltschaft und Verteidigung) aufzuteilen und die Anklage gegen bzw. die Verteidigung von Daniel B. vorzubereiten.

\hypertarget{prosecution}{%
\section{Die Staatsanwaltschaft}\label{prosecution}}

Die Staatsanwaltschaft besteht aus den Teilnehmer*innen des Seminars mit den folgenden Initialen: \texttt{A.\ C.}, \texttt{A.\ K.}, \texttt{A.\ W.}, \texttt{A.\ B.}, \texttt{A.\ G.}, \texttt{D.\ M.}, \texttt{D.\ K.}, \texttt{E.\ L.} und \texttt{J.\ H.}

Die Staatsanwaltschaft wird gebeten alle in dem Kapitel 3 \protect\hyperlink{prosecution-evidence}{\emph{Beweismittel der Staatsanwaltschaft}} zur Verfügung stehenden Materialien und Ressourcen zu nutzen, um die Anklage gegen Daniel B. vorzubereiten. Ihr obliegt es zweifelsfrei darzulegen, dass Daniel B. zum Zeitpunkt des Tat \textbf{schuldfähig} war.

Die Teilnehmer*in mit den Initialen \texttt{A.\ G.} wird im Rahmen der Verhandlung die Rolle des*der \textbf{ersten Sachverständigen} übernehmen. In seiner*ihrer Rolle kann und soll \texttt{A.\ G.} sowohl von der Staatsanwaltschaft als auch von der Verteidigung bzgl. des Themenkomplexes ``Freier Wille'' befragt werden.

Des Weiteren bitten die ehrenwerten Richter*innen die Staatsanwaltschaft ein \textbf{Plädoyer} (Schlusswort) vorzubereiten. Zu Dokumentationszwecken sollte besagtes Schlusswort (zumindest in Stichworten) \textbf{niedergeschrieben} werden.

\hypertarget{defence}{%
\section{Die Verteidigung}\label{defence}}

Die Verteidigung besteht aus den Teilnehmer*innen des Seminars mit den folgenden Initialen: \texttt{J.\ M.}, \texttt{K.\ G.}, \texttt{K.\ H.}, \texttt{L.\ H.}, \texttt{L.\ S.}, \texttt{L.\ G.}, \texttt{M.\ P.}, \texttt{M.\ L.}, \texttt{M.\ H.} und \texttt{V.\ B.}

Die Verteidigung wird hingegen gebeten alle in dem Kapitel \protect\hyperlink{defence-evidence}{\emph{Beweismittel der Verteidigung}} zur Verfügung stehenden Materialien und Ressourcen zu nutzen, um die Verteidigung von Daniel B. vorzubereiten. Ihr obliegt es zweifelsfrei darzulegen, dass Daniel B. zum Zeitpunkt des Tat \textbf{nicht schuldfähig} war.

Die Teilnehmer*in mit den Initialen \texttt{L.\ G.} wird im Rahmen der Verhandlung die Rolle des*der \textbf{zweiten Sachverständigen} übernehmen. In seiner*ihrer Rolle kann und soll \texttt{L.\ G.} sowohl von der Verteidigung als auch von der Staatsanwaltschaft bzgl. des Themenkomplexes ``Freier Wille'' befragt werden.

Des Weiteren bitten die ehrenwerten Richter*innen die Verteidigung ein \textbf{Plädoyer} (Schlusswort) vorzubereiten. Zu Dokumentationszwecken sollte besagtes Schlusswort (zumindest in Stichworten) \textbf{niedergeschrieben} werden.

\begin{longtable}[]{@{}
  >{\raggedright\arraybackslash}p{(\columnwidth - 4\tabcolsep) * \real{0.2603}}
  >{\raggedright\arraybackslash}p{(\columnwidth - 4\tabcolsep) * \real{0.3699}}
  >{\raggedright\arraybackslash}p{(\columnwidth - 4\tabcolsep) * \real{0.3699}}@{}}
\caption{Staatsanwaltschaft und Verteidigung im Überblick}\tabularnewline
\toprule()
\begin{minipage}[b]{\linewidth}\raggedright
\end{minipage} & \begin{minipage}[b]{\linewidth}\raggedright
Staatsanwaltschaft
\end{minipage} & \begin{minipage}[b]{\linewidth}\raggedright
Verteidigung
\end{minipage} \\
\midrule()
\endfirsthead
\toprule()
\begin{minipage}[b]{\linewidth}\raggedright
\end{minipage} & \begin{minipage}[b]{\linewidth}\raggedright
Staatsanwaltschaft
\end{minipage} & \begin{minipage}[b]{\linewidth}\raggedright
Verteidigung
\end{minipage} \\
\midrule()
\endhead
\textbf{Mitglieder*innen} & \texttt{A.\ C.}, \texttt{A.\ K.}, \texttt{A.\ W.}, \texttt{A.\ B.}, \texttt{A.\ G.}, \texttt{D.\ M.}, \texttt{D.\ K.}, \texttt{E.\ L.} und \texttt{J.\ H.} & \texttt{J.\ M.}, \texttt{K.\ G.}, \texttt{K.\ H.}, \texttt{L.\ H.}, \texttt{L.\ S.}, \texttt{L.\ G.}, \texttt{M.\ P.}, \texttt{M.\ L.}, \texttt{M.\ H.} und \texttt{V.\ B.} \\
\textbf{Sachverständige*r} & \texttt{A.\ G.} & \texttt{L.\ G.} \\
\bottomrule()
\end{longtable}

\hypertarget{timetable}{%
\section{Der Ablauf der Verhandlung}\label{timetable}}

\hypertarget{timetable-sa}{%
\subsection{Samstag, den 14.01.2023}\label{timetable-sa}}

\begin{longtable}[]{@{}
  >{\raggedright\arraybackslash}p{(\columnwidth - 4\tabcolsep) * \real{0.2500}}
  >{\raggedright\arraybackslash}p{(\columnwidth - 4\tabcolsep) * \real{0.2917}}
  >{\raggedright\arraybackslash}p{(\columnwidth - 4\tabcolsep) * \real{0.4583}}@{}}
\caption{Ablauf der Verhandlung am 14.01.2023}\tabularnewline
\toprule()
\begin{minipage}[b]{\linewidth}\raggedright
Uhrzeit
\end{minipage} & \begin{minipage}[b]{\linewidth}\raggedright
Titel des Elements
\end{minipage} & \begin{minipage}[b]{\linewidth}\raggedright
Beschreibung
\end{minipage} \\
\midrule()
\endfirsthead
\toprule()
\begin{minipage}[b]{\linewidth}\raggedright
Uhrzeit
\end{minipage} & \begin{minipage}[b]{\linewidth}\raggedright
Titel des Elements
\end{minipage} & \begin{minipage}[b]{\linewidth}\raggedright
Beschreibung
\end{minipage} \\
\midrule()
\endhead
16:30 Uhr & \textbf{Impulsreferat} & Einführung in das Thema und Erläuterung des Ablaufs der Sitzung \\
16:40 Uhr & \textbf{Vorbereitung der Verhandlung} & Sichtung und Diskussion der Materialien innerhalb der Gruppen \textbf{Staatsanwaltschaft} und \textbf{Verteidigung}. \\
17:10 Uhr & \textbf{Eröffnung der Gerichtsverhandlung} & Verlesung der Anklageschrift durch die Richter*innen und Beginn der Verhandlung. \\
17:15 Uhr & \textbf{Beweisanträge der Staatsanwaltschaft} & Darlegung der Argumente bzw. Beweismittel \textbf{für} die Schuldfähigkeit von Daniel B. (inkl. der Befragung des Sachverständigen \texttt{A.\ G.}). \\
17:30 Uhr & \textbf{Beweisanträge der Verteidigung} & Darlegung der Argumente bzw. Beweismittel \textbf{gegen} die Schuldfähigkeit von Daniel B. (inkl. der Befragung des Sachverständigen \texttt{L.\ G.}). \\
17:45 Uhr & \textbf{Plädoyer der Staatsanwaltschaft} & Schlusswort der Staatsanwaltschaft. \\
17:50 Uhr & \textbf{Plädoyer der Verteidigung} & Schlusswort der Verteidigung. \\
17:55 Uhr & \textbf{Ende des ersten Verhandlungstages} & \\
\bottomrule()
\end{longtable}

\hypertarget{timetable-sun}{%
\subsection{Sonntag, den 15.01.2023}\label{timetable-sun}}

\begin{longtable}[]{@{}
  >{\raggedright\arraybackslash}p{(\columnwidth - 4\tabcolsep) * \real{0.2500}}
  >{\raggedright\arraybackslash}p{(\columnwidth - 4\tabcolsep) * \real{0.3056}}
  >{\raggedright\arraybackslash}p{(\columnwidth - 4\tabcolsep) * \real{0.4444}}@{}}
\caption{Ablauf der Verhandlung am 15.01.2023}\tabularnewline
\toprule()
\begin{minipage}[b]{\linewidth}\raggedright
Uhrzeit
\end{minipage} & \begin{minipage}[b]{\linewidth}\raggedright
Titel des Elements
\end{minipage} & \begin{minipage}[b]{\linewidth}\raggedright
Beschreibung
\end{minipage} \\
\midrule()
\endfirsthead
\toprule()
\begin{minipage}[b]{\linewidth}\raggedright
Uhrzeit
\end{minipage} & \begin{minipage}[b]{\linewidth}\raggedright
Titel des Elements
\end{minipage} & \begin{minipage}[b]{\linewidth}\raggedright
Beschreibung
\end{minipage} \\
\midrule()
\endhead
09:30 Uhr & \textbf{Urteilsverkündung} & Verkündung und Begründung des Urteils durch die Richter*innen. \\
09:40 Uhr & \textbf{Reflexion der Sitzung} & Kritische Reflexion der Sitzung im Rahmen einer freien Gruppendiskussion \\
10:00 Uhr & \textbf{Ende der Verhandlung} & \\
\bottomrule()
\end{longtable}

\hypertarget{prosecution-evidence}{%
\chapter{Beweismittel der Staatsanwaltschaft}\label{prosecution-evidence}}

Eure Position als Staatsanwaltschaft ist die folgende:

\begin{quote}
\emph{Daniel B. war zum Zeitpunkt der Tat schuldfähig und ist daher schuldig zu sprechen.}
\end{quote}

Eure Position basiert auf der Annahme, dass Daniel B. einen \textbf{freien Willen} besitzt und sich \textbf{bewusst} dazu entschieden hat Sigmund F. zu ermorden.

Nutzt bitte die folgenden Textausschnitte, um eine sinnvolle Argumentation aufzubauen. Selbstverständlich könnt ihr zur Vorbereitung weitere seriöse Quellen eurer Wahl nutzen.

Die*der Sachverständige*r \texttt{A.\ G.} soll im Rahmen der Verhandlung sowohl von euch als Staatsanwaltschaft als auch von der Verteidigung befragt werden. Dementsprechend ist es wichtig das Sachverständige*r \texttt{A.\ G.} einen sehr guten Überblick über die von euch verwendeten Beweismittel hat.

\hypertarget{pr-ev1}{%
\section{\#1 Haben wir einen freien Willen?}\label{pr-ev1}}

\textbf{Auszug aus dem Aufsatz ``Haben wir einen freien Willen?'' von Benjamin Libet \citeyearpar{Libet2004}}

\begin{quote}
„Ich habe mich dieser Frage auf experimentelle Weise genähert. Freien Willenshandlungen geht eine spezifische elektrische Veränderung im Gehirn voraus (das ›Bereitschaftspotentiak‹, BP), das 550 ms vor der Handlung einsetzt. Menschliche Versuchspersonen wurden sich der Handlungsintention 350-400 ms nach Beginn von BP bewußt, aber 200 ms vor der motorischen Handlung. Der Willensprozeß wird daher unbewußt eingeleitet. Aber die Bewußtseinsfunktion kann den Ausgang immer noch steuern; sie kann die Handlung durch ein Veto verbieten. Willensfreiheit ist daher nicht ausgeschlossen. Diese Befunde stellen Beschränkungen für mögliche Ansichten darüber dar, wie der freie Wille funktionieren könnte; er würde eine Willenshandlung nicht einleiten, würde aber den Vollzug der Handlung steuern.‟ (S.268)
\end{quote}

\hypertarget{pr-ev2}{%
\section{\#2 Warum noch debattieren?}\label{pr-ev2}}

\textbf{Auszug aus dem Aufsatz ``Warum noch debattieren? Determinismus als Diskurskiller'' von Gerhard Kaiser \citeyearpar{Kaiser2004}}

\begin{quote}
„Aber derselbe Wolf Singer, der von Komplementarität spricht, verwendet fortgesetzt grenzüberschreitend das Kausalitätsschema (S. 29), nennt Gedanken »Folge neuronaler Prozesse« (S. 15) und weitet die deterministische Position sogar dahin aus, daß »auch die kulturelle Umwelt determiniert« (S. 23). Die strikte Determination der geistigen Zustände und Aktivitäten des Menschen letztendlich durch die neuronalen Gegebenheiten gewinnt so die Qualität eines Glaubenssatzes.

Es ist ein Glaube, der weit über alles hinausgreift, was sich experimentell nachweisen läßt. Wie Singer wiederholt feststellt, gibt es enorme qualitative Sprünge von der immer noch relativ einfachen neuronalen Bearbeitung von Sinneseindrücken bis hin etwa zum Verstehen von Sprache (als nicht nur Sinneseindruck, sondern Mitteilung). Ahnlich weit ist der Sprung von der Verknüpfung neuronaler Reizzustände mit einfachen Handlungsentscheidungen etwa für die Öffnung einer rechten oder linken Tür, wie sie auch Primaten treffen, bis hin zur neuronalen Entsprechung für Wallensteins Erwägung: »Wär's möglich? Könnt' ich nicht mehr, wie ich wollte?« (»Wallensteins Tode«, IV, 4) oder zu der Überlegung: Determinieren neuronale Erregungszustände mein Denken und Handeln? Abgesehen davon wäre auch bei relativ einfachen experimentellen Ergebnissen in einem derartig sensiblen Bereich wie der Hirnforschung das Grundprinzip der Elementarteilchen-Physik zu bedenken: daß nämlich der beobachtete Vorgang durch den Vorgang der Beobachtung beeinflußt wird, daß dem Experiment also die Beobachterposition eingeschrieben ist.

Und nicht nur das. Naturwissenschaftliche Experimente gliedern ihr Beobachtungsfeld ab ovo aus der Lebenswelt des Beobachters aus, reduzieren ihren Gegenstand auf das Wiederholbare und Quantifizierbare und versachlichen den Experimentator selbst zur neutralen Beobachterinstanz.‟ (S. 263-264)
\end{quote}

\hypertarget{pr-ev3}{%
\section{\#3 Infektion des Geistes}\label{pr-ev3}}

\textbf{Auszüge aus dem Aufsatz ``Infektion des Geistes. Über philosophische Kategorienfehler'' von Gerd Kempermann \citeyearpar{Kempermann2004}}

\begin{quote}
„Ohne Gehirn kein menschlicher Geist. Es ist ebenfalls eine Alltagserfahrung, daß der Wille so frei nicht ist. Die Psychologie der Politik und des Gesundheitsverhaltens sind Beispiele, wie Menschen anders handeln, als sie könnten und »eigentlich« wollen.‟ (S. 235)
\end{quote}

\begin{quote}
„Der Mensch sei (je nach Ideologie) zu 10, 50 oder 90 Prozent durch die Umwelt und der Rest auf hundert Prozent durch seine Gene bestimmt. Das ist Unsinn. Der Mensch ist ganz durch seine Gene und ganz durch seine Umwelt bestimmt. Diese Wechselwirkung ist wörtlich zu nehmen.‟ (S. 235)
\end{quote}

\begin{quote}
„Der freie Wille gehört wie die Menschenwürde in die Karegorie der Konstrukte, die Zuschreibungen sind. Freier Wille bleibt schon in der Eigenwahrnehmung unscharf. Man mag prinzipiell viel auf die eigene Autonomie halten, um sie doch in schwierigen Situationen, nicht nur in Strafverfahren, bereitwillig über Bord zu werfen und auf biologische Entlastungsgründe zurückzugreifen.‟ (S. 236)
\end{quote}

\hypertarget{pr-ev4}{%
\section{\#4 Ick bün all da}\label{pr-ev4}}

\textbf{Auszüge aus dem Aufsatz ``Ick bün all da. Ein neuronales Erregungsmuster'' von Reinhard Brandt \citeyearpar{Brandt2004}}

\begin{quote}
„» Wie sünd all da!« Mit ihren raffinierten Horchgeräten und Kernspintomographen hören die Forscher das homerische Gelächter der Zellen über den verwirrten Geist und den Willen der Menschen, die sich frei und selbständig dünken und doch nur ausführen, was im grauen Netzwerk der Zellen zuvor festgelegt wurde. Der Geist gleicht der Fliege, die auf einem Wagenrad sitzt und sich einbildet, das Rad zu bewegen.‟ (S. 171)
\end{quote}

\begin{quote}
„Und das Gefühl der Freiheit, das unweigerlich unsere Entschlüsse und die ihnen folgenden Handlungen begleitet, auch dieses unvermeidliche Gefühl ist wohl nichts anderes als eine Illusion, die aus der Retrospektive auch meistens verschwindet: Man hätte damals eben doch nicht anders handeln können. Das Freiheitsgefühl scheint nur der blinde Fleck im Auge zu sein, der uns die determinierenden Ursachen nicht sehen läßt. Im übrigen verhalten wir uns grundsätzlich so, daß wir nicht an die Freiheit der Menschen glauben; wenn das freie Wahlvolk anders stimmt oder der freie Nachbar anders handelt, als erwartet, so fragen wir nach den Ursachen und Motiven wie beim Ausfall von Gas und Strom oder beim auffälligen Verhalten von Hunden und Katzen. Du hättest anders handeln können, das sagen wir bevorzugt, wenn es um Schuldzuweisungen geht, aber sonst sind wir »Deterministen«: Von dem und dem war nichts anderes zu erwarten; Sokrates lügt eben nicht, der kann nicht anders.‟ (S. 171-172)
\end{quote}

\begin{quote}
„Wir sehen Optisches, wir hören Akustisches, wir ertasten Haptisches - aber was das jeweils für Dinge sind, die wir da sehen und hören und ertasten, darüber können uns die Sinne nicht mehr belehren. Wir meinen, wir würden einen im Wasser gebrochenen Stab sehen, der dann doch gerade ist. Aber das ist eine Illusion des Skeptikers: Was wir völlig richtig sehen, ist eine graue gebrochene breitere Linie; daß es ein Stab ist, gar ein gebrochener Stab, dazu schweigt der Sehsinn pflichtgemäß, denn von Stäben weiß er nichts. Was ein Stab ist, sagt uns der Verstand.

Blickt der Neurologe ins Feld, so sieht er keinen Igel und keinen Hasen, sondern bestimmte Formen und Farben und Bewegungen, die er, ein igel- und hasengewohnter Abendländer, richtig erkennt. Vielleicht hielt er den Hasen zuerst für ein Kaninchen und ließ sich durch einen Spezialisten, etwa einen Jäger, dankbar korrigieren: \emph{Sehen} läßt sich nicht, was ein Hase und was ein Kaninchen ist. Im Labor sieht er etwas Graues, das sich weich und feucht anfühlt und stumm ist; als wissenschaftlicher Spezialist erkennt er dieses Etwas als Gehirn; mit seinen Apparaten identifiziert er Synapsen und Zellen, immer im Zusammenspiel von Sinnesinformationen und hochspezialisierten Erkenntnisleistungen. Seine Berichte über die Ergebnisse verzerren, aber sie vereinfachen auch die Situation durch die Benutzung der schon genannten Illusion: Er tut so, als könnte er die Zeilen als Zellen und die Vernetzungen als Vernetzungen \emph{sehen} - davon kann zwar die Rede sein, der Sache nach ist diese Redeform jedoch unhaltbar: Zellen und Vernetzungen kann man so wenig sehen wie Telefone und Computer, Stäbe und Sonnen und Monde.‟ (S. 173)
\end{quote}

\begin{quote}
„Jede mitteilbare wahre oder falsche Erkenntnis setzt sich aus Urteilen zusammen wie z. B. »Alles Wissen, über das ein Gehirn verfügt, residiert in seiner funktionellen Architektur«; ein Urteil ist im einfachsten Fall die Einheitsverknüpfung von Subjekt und Prädikat, wobei diese Verknüpfung notwendig entweder bejahend oder verneinend ist (hier also: » Nicht alles Wissen, {[}\ldots{]}« oder »{[}\ldots{]} residiert nicht in seiner Architektur«). Die Verneinung vereint Subjekt und Prädikat im Urteil und behauptet zugleich deren Trennung. Die gegen die Vorstellung vom Primat der Materie oder des Gehirns gegenüber dem Geist gerichtete These lautet: In keiner Gehirnzelle und in keiner Synapse hat man und wird man das Äquivalent eines Urteils, besonders keine Verneinung entdecken. Wer je im Gehirn eine Verneinung auffindet, dem verpfände ich alle Synapsen, die an dieser Zeile beteiligt sind (beim Milliardenaufkommen wird das ja zu verkraften sein). Solange eine Urteils- oder Erkenntnisbildung und besonders eine Verneinung nicht entdeckt wurden, läßt sich der Geist nicht auf noch so dynamische und demokratisch vernetzte Prozesse des Gehirns zurückführen. Sie bieten notwendige, aber keine hinreichenden Bedingungen für den Geist, der nein sagen kann und mit seinem ersten Nein ins Dasein sprang.‟ (S. 175)
\end{quote}

\hypertarget{pr-more}{%
\section{Weitere Quellen}\label{pr-more}}

\begin{itemize}
\item
  Aufsatz „Gründe zählen. Über einige Schwierigkeiten des Bionaturalismus‟ von Lutz Wingert \citeyearpar{Lutz2004}
\item
  Aufsatz „Verschaltungen legen uns fest: Wir sollten aufhören, von Freiheit zu sprechen‟ von Wolf Singer \citeyearpar{Singer2004}
\item
  Aufsatz „Der entlarvte Ruck. Was sagt Kant den Gehirnforschern‟ von Otfried Höffe \citeyearpar{Höffe2004}
\end{itemize}

\hypertarget{defence-evidence}{%
\chapter{Beweismittel der Verteidigung}\label{defence-evidence}}

Eure Position als Verteidigung ist die folgende:

\begin{quote}
\emph{Daniel B. war zum Zeitpunkt der Tat nicht schuldfähig und ist daher frei zu sprechen.}
\end{quote}

Eure Position basiert auf der Annahme, dass Daniel B. \textbf{keinen} \textbf{freien Willen} besitzt und sich daher \textbf{nicht} \textbf{bewusst} dazu entscheiden konnte Sigmund F. zu ermorden.

Nutzt bitte die folgenden Textausschnitte, um eine sinnvolle Argumentation aufzubauen. Selbstverständlich könnt ihr zur Vorbereitung weitere seriöse Quellen eurer Wahl nutzen.

Die*der Sachverständige*r \texttt{L.\ G.} soll im Rahmen der Verhandlung sowohl von euch als Verteidigung als auch von der Staatsanwaltschaft befragt werden. Dementsprechend ist es wichtig das Sachverständige*r \texttt{L.\ G.} einen sehr guten Überblick über die von euch verwendeten Beweismittel hat.

\hypertarget{def-ev1}{%
\section{\#1 Haben wir einen freien Willen?}\label{def-ev1}}

\textbf{Auszug aus dem Aufsatz ``Haben wir einen freien Willen?'' von Benjamin Libet \citeyearpar{Libet2004}}

\begin{quote}
„Ich habe mich dieser Frage auf experimentelle Weise genähert. Freien Willenshandlungen geht eine spezifische elektrische Veränderung im Gehirn voraus (das ›Bereitschaftspotentiak‹, BP), das 550 ms vor der Handlung einsetzt. Menschliche Versuchspersonen wurden sich der Handlungsintention 350-400 ms nach Beginn von BP bewußt, aber 200 ms vor der motorischen Handlung. Der Willensprozeß wird daher unbewußt eingeleitet.‟ (S.268)
\end{quote}

\hypertarget{def-ev2}{%
\section{\#2 Warum noch debattieren?}\label{def-ev2}}

\textbf{Auszüge aus dem Aufsatz ``Warum noch debattieren? Determinismus als Diskurskiller'' von Gerhard Kaiser \citeyearpar{Kaiser2004}}

\begin{quote}
„Aber derselbe Wolf Singer, der von Komplementarität spricht, verwendet fortgesetzt grenzüberschreitend das Kausalitätsschema (S. 29), nennt Gedanken »Folge neuronaler Prozesse« (S. 15) und weitet die deterministische Position sogar dahin aus, daß »auch die kulturelle Umwelt determiniert« (S. 23).‟ (S. 263)
\end{quote}

\begin{quote}
„Willensfreiheit ist relativ - menschliche Freiheit ist nie so gedacht worden, daß dem Menschen alle je denkbaren Optionen auch offenstehen. Sie findet statt innerhalb eines individuellen und historischen Ermöglichungs- und Bedingungsrahmens. Und sie ist, jedenfalls auf der Stufe des reflexiven Bewußtseins, ein Spezifikum des Menschen.‟ (S. 265)
\end{quote}

\hypertarget{def-ev3}{%
\section{\#3 Infektion des Geistes}\label{def-ev3}}

\textbf{Auszüge aus dem Aufsatz ``Infektion des Geistes. Über philosophische Kategorienfehler'' von Gerd Kempermann \citeyearpar{Kempermann2004}}

\begin{quote}
„Der freie Wille gehört wie die Menschenwürde in die Karegorie der Konstrukte, die Zuschreibungen sind. Freier Wille bleibt schon in der Eigenwahrnehmung unscharf. Man mag prinzipiell viel auf die eigene Autonomie halten, um sie doch in schwierigen Situationen, nicht nur in Strafverfahren, bereitwillig über Bord zu werfen und auf biologische Entlastungsgründe zurückzugreifen. Wir sind uns unserer Natur und der Abhängigkeit von ihr durchaus bewußt. Aber das befreit uns nicht davon, verantwortlich zu handeln. Denn es ist ja nicht nur Natur in uns.‟ (S. 236)
\end{quote}

\begin{quote}
„Wenn man in Zusammenhängen, die wir bisher der Psyche zugeschrieben haben, die Erkenntnisse der Biologie zuläßt, brechen nicht gleich die moralischen Fundamente des Abendlandes weg. Durch Berücksichtigung der »Natur des Menschen« gelingt es oft erst, zu wirklich menschlichen Lösungen zu gelangen. Die forensische Psychiatrie beschäftigt sich seit langem mit diesen Fragen, und auch sie lernt nicht nur durch soziopsychologische Forschung, sondern auch durch Neurobiologie dazu.‟ (S. 237)
\end{quote}

\begin{quote}
„Die bestimmende Interaktion zwischen Natur und persönlicher Lebensgeschichte gilt selbstverständlich nicht nur für die Psychiatrie. Bei Herzinfarkten, Typ-II-Diabetes und bestimmten Krebsformen ist das Zusammenspiel von Disposition und Lebensführung Allgemeingut geworden. Was ist Lebensführung anderes als der persönliche, »freie« Umgang mit Disposition, das verantwortliche Leben der eigenen Natur?‟ (S. 238)
\end{quote}

\begin{quote}
„Eine Ethik können nur Menschen haben. Biologische Fakten anzuerkennen heißt nicht, der Verantwortung ledig zu sein, eine solche Ethik zu entwickeln.‟ (S. 239)
\end{quote}

\hypertarget{def-ev4}{%
\section{\#4 Ick bün all da}\label{def-ev4}}

\textbf{Auszüge aus dem Aufsatz ``Ick bün all da. Ein neuronales Erregungsmuster'' von Reinhard Brandt \citeyearpar{Brandt2004}}

\begin{quote}
„» Wie sünd all da!« Mit ihren raffinierten Horchgeräten und Kernspintomographen hören die Forscher das homerische Gelächter der Zellen über den verwirrten Geist und den Willen der Menschen, die sich frei und selbständig dünken und doch nur ausführen, was im grauen Netzwerk der Zellen zuvor festgelegt wurde. Der Geist gleicht der Fliege, die auf einem Wagenrad sitzt und sich einbildet, das Rad zu bewegen.‟ (S. 171)
\end{quote}

\begin{quote}
„Und das Gefühl der Freiheit, das unweigerlich unsere Entschlüsse und die ihnen folgenden Handlungen begleitet, auch dieses unvermeidliche Gefühl ist wohl nichts anderes als eine Illusion, die aus der Retrospektive auch meistens verschwindet: Man hätte damals eben doch nicht anders handeln können. Das Freiheitsgefühl scheint nur der blinde Fleck im Auge zu sein, der uns die determinierenden Ursachen nicht sehen läßt. Im übrigen verhalten wir uns grundsätzlich so, daß wir nicht an die Freiheit der Menschen glauben; wenn das freie Wahlvolk anders stimmt oder der freie Nachbar anders handelt, als erwartet, so fragen wir nach den Ursachen und Motiven wie beim Ausfall von Gas und Strom oder beim auffälligen Verhalten von Hunden und Katzen. Du hättest anders handeln können, das sagen wir bevorzugt, wenn es um Schuldzuweisungen geht, aber sonst sind wir »Deterministen«: Von dem und dem war nichts anderes zu erwarten; Sokrates lügt eben nicht, der kann nicht anders.‟ (S. 171-172)
\end{quote}

\begin{quote}
„Wir sehen Optisches, wir hören Akustisches, wir ertasten Haptisches - aber was das jeweils für Dinge sind, die wir da sehen und hören und ertasten, darüber können uns die Sinne nicht mehr belehren. Wir meinen, wir würden einen im Wasser gebrochenen Stab sehen, der dann doch gerade ist. Aber das ist eine Illusion des Skeptikers: Was wir völlig richtig sehen, ist eine graue gebrochene breitere Linie; daß es ein Stab ist, gar ein gebrochener Stab, dazu schweigt der Sehsinn pflichtgemäß, denn von Stäben weiß er nichts. Was ein Stab ist, sagt uns der Verstand.

Blickt der Neurologe ins Feld, so sieht er keinen Igel und keinen Hasen, sondern bestimmte Formen und Farben und Bewegungen, die er, ein igel- und hasengewohnter Abendländer, richtig erkennt. Vielleicht hielt er den Hasen zuerst für ein Kaninchen und ließ sich durch einen Spezialisten, etwa einen Jäger, dankbar korrigieren: \emph{Sehen} läßt sich nicht, was ein Hase und was ein Kaninchen ist. Im Labor sieht er etwas Graues, das sich weich und feucht anfühlt und stumm ist; als wissenschaftlicher Spezialist erkennt er dieses Etwas als Gehirn; mit seinen Apparaten identifiziert er Synapsen und Zellen, immer im Zusammenspiel von Sinnesinformationen und hochspezialisierten Erkenntnisleistungen. Seine Berichte über die Ergebnisse verzerren, aber sie vereinfachen auch die Situation durch die Benutzung der schon genannten Illusion: Er tut so, als könnte er die Zeilen als Zellen und die Vernetzungen als Vernetzungen \emph{sehen} - davon kann zwar die Rede sein, der Sache nach ist diese Redeform jedoch unhaltbar: Zellen und Vernetzungen kann man so wenig sehen wie Telefone und Computer, Stäbe und Sonnen und Monde.‟ (S. 173)
\end{quote}

\begin{quote}
„Jede mitteilbare wahre oder falsche Erkenntnis setzt sich aus Urteilen zusammen wie z. B. »Alles Wissen, über das ein Gehirn verfügt, residiert in seiner funktionellen Architektur«; ein Urteil ist im einfachsten Fall die Einheitsverknüpfung von Subjekt und Prädikat, wobei diese Verknüpfung notwendig entweder bejahend oder verneinend ist (hier also: » Nicht alles Wissen, {[}\ldots{]}« oder »{[}\ldots{]} residiert nicht in seiner Architektur«). Die Verneinung vereint Subjekt und Prädikat im Urteil und behauptet zugleich deren Trennung. Die gegen die Vorstellung vom Primat der Materie oder des Gehirns gegenüber dem Geist gerichtete These lautet: In keiner Gehirnzelle und in keiner Synapse hat man und wird man das Äquivalent eines Urteils, besonders keine Verneinung entdecken. Wer je im Gehirn eine Verneinung auffindet, dem verpfände ich alle Synapsen, die an dieser Zeile beteiligt sind (beim Milliardenaufkommen wird das ja zu verkraften sein). Solange eine Urteils- oder Erkenntnisbildung und besonders eine Verneinung nicht entdeckt wurden, läßt sich der Geist nicht auf noch so dynamische und demokratisch vernetzte Prozesse des Gehirns zurückführen. Sie bieten notwendige, aber keine hinreichenden Bedingungen für den Geist, der nein sagen kann und mit seinem ersten Nein ins Dasein sprang.‟ (S. 175)
\end{quote}

\hypertarget{you-dont-have-free-will-but-dont-worry}{%
\section{\#5 You don't have free will, but don't worry}\label{you-dont-have-free-will-but-dont-worry}}

Ein Video von \href{https://sabinehossenfelder.com}{Dr.~Sabine Hossenfelder} (theoretische Physikerin), in dem sie erklärt warum die Idee des freien Willen mit den derzeit bekannten Naturgesetzen unvereinbar ist \citep{Hossenfelder2020}.

\href{https://www.youtube.com/watch?v=zpU_e3jh_FY}{Video: ``You don't have free will, but don't worry.''}

\hypertarget{def-more}{%
\section{Weitere Quellen}\label{def-more}}

\begin{itemize}
\item
  Aufsatz „Gründe zählen. Über einige Schwierigkeiten des Bionaturalismus‟ von Lutz Wingert \citeyearpar{Lutz2004}
\item
  Aufsatz „Verschaltungen legen uns fest: Wir sollten aufhören, von Freiheit zu sprechen‟ von Wolf Singer \citeyearpar{Singer2004}
\item
  Aufsatz „Der entlarvte Ruck. Was sagt Kant den Gehirnforschern‟ von Otfried Höffe \citeyearpar{Höffe2004}
\end{itemize}

\hypertarget{results}{%
\chapter{Ergebnisse der Seminarsitzung}\label{results}}

\hypertarget{pr-plea}{%
\section{Plädoyer der Staatsanwaltschaft}\label{pr-plea}}

\hypertarget{def-plea}{%
\section{Plädoyer der Verteidigung}\label{def-plea}}

\hypertarget{sentencing}{%
\section{Urteilsverkündung}\label{sentencing}}

\hypertarget{reflection}{%
\section{Reflexion der Sitzung}\label{reflection}}

  \bibliography{refs.bib}

\end{document}
