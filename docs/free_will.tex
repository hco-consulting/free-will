% Options for packages loaded elsewhere
\PassOptionsToPackage{unicode}{hyperref}
\PassOptionsToPackage{hyphens}{url}
\PassOptionsToPackage{dvipsnames,svgnames,x11names}{xcolor}
%
\documentclass[
  a4paper,
]{report}
\usepackage{amsmath,amssymb}
\usepackage{lmodern}
\usepackage{iftex}
\ifPDFTeX
  \usepackage[T1]{fontenc}
  \usepackage[utf8]{inputenc}
  \usepackage{textcomp} % provide euro and other symbols
\else % if luatex or xetex
  \usepackage{unicode-math}
  \defaultfontfeatures{Scale=MatchLowercase}
  \defaultfontfeatures[\rmfamily]{Ligatures=TeX,Scale=1}
\fi
% Use upquote if available, for straight quotes in verbatim environments
\IfFileExists{upquote.sty}{\usepackage{upquote}}{}
\IfFileExists{microtype.sty}{% use microtype if available
  \usepackage[]{microtype}
  \UseMicrotypeSet[protrusion]{basicmath} % disable protrusion for tt fonts
}{}
\makeatletter
\@ifundefined{KOMAClassName}{% if non-KOMA class
  \IfFileExists{parskip.sty}{%
    \usepackage{parskip}
  }{% else
    \setlength{\parindent}{0pt}
    \setlength{\parskip}{6pt plus 2pt minus 1pt}}
}{% if KOMA class
  \KOMAoptions{parskip=half}}
\makeatother
\usepackage{xcolor}
\usepackage[top=20mm,left=30mm,heightrounded]{geometry}
\usepackage{longtable,booktabs,array}
\usepackage{calc} % for calculating minipage widths
% Correct order of tables after \paragraph or \subparagraph
\usepackage{etoolbox}
\makeatletter
\patchcmd\longtable{\par}{\if@noskipsec\mbox{}\fi\par}{}{}
\makeatother
% Allow footnotes in longtable head/foot
\IfFileExists{footnotehyper.sty}{\usepackage{footnotehyper}}{\usepackage{footnote}}
\makesavenoteenv{longtable}
\usepackage{graphicx}
\makeatletter
\def\maxwidth{\ifdim\Gin@nat@width>\linewidth\linewidth\else\Gin@nat@width\fi}
\def\maxheight{\ifdim\Gin@nat@height>\textheight\textheight\else\Gin@nat@height\fi}
\makeatother
% Scale images if necessary, so that they will not overflow the page
% margins by default, and it is still possible to overwrite the defaults
% using explicit options in \includegraphics[width, height, ...]{}
\setkeys{Gin}{width=\maxwidth,height=\maxheight,keepaspectratio}
% Set default figure placement to htbp
\makeatletter
\def\fps@figure{htbp}
\makeatother
\setlength{\emergencystretch}{3em} % prevent overfull lines
\providecommand{\tightlist}{%
  \setlength{\itemsep}{0pt}\setlength{\parskip}{0pt}}
\setcounter{secnumdepth}{5}
\ifLuaTeX
\usepackage[bidi=basic]{babel}
\else
\usepackage[bidi=default]{babel}
\fi
\babelprovide[main,import]{ngerman}
% get rid of language-specific shorthands (see #6817):
\let\LanguageShortHands\languageshorthands
\def\languageshorthands#1{}
\ifLuaTeX
  \usepackage{selnolig}  % disable illegal ligatures
\fi
\usepackage[]{natbib}
\bibliographystyle{plainnat}
\IfFileExists{bookmark.sty}{\usepackage{bookmark}}{\usepackage{hyperref}}
\IfFileExists{xurl.sty}{\usepackage{xurl}}{} % add URL line breaks if available
\urlstyle{same} % disable monospaced font for URLs
\hypersetup{
  pdftitle={Freier Wille als Illusion oder Notwendigkeit},
  pdfauthor={Emma Erhard, Universität Konstanz; Hamilkar Constantin Oueslati, Universität Konstanz; Naima Steimel, Universität Konstanz},
  pdflang={de},
  colorlinks=true,
  linkcolor={Maroon},
  filecolor={Maroon},
  citecolor={Blue},
  urlcolor={Blue},
  pdfcreator={LaTeX via pandoc}}

\title{Freier Wille als Illusion oder Notwendigkeit}
\author{Emma Erhard, Universität Konstanz \and Hamilkar Constantin Oueslati, Universität Konstanz \and Naima Steimel, Universität Konstanz}
\date{14.01.2023}

\begin{document}
\maketitle

\renewcommand*\contentsname{Inhaltsverzeichnis}
{
\hypersetup{linkcolor=}
\setcounter{tocdepth}{3}
\tableofcontents
}
\hypertarget{preface}{%
\chapter*{Vorwort}\label{preface}}
\addcontentsline{toc}{chapter}{Vorwort}

\hypertarget{about}{%
\section*{Über dieses Webbook}\label{about}}
\addcontentsline{toc}{section}{Über dieses Webbook}

Dieses Webbook wurde im Rahmen des Seminars ``Von Freiheit und Notwendigkeit'' (WS 2022/2023 - Universität Konstanz) erstellt. Es enthält alle Informationen und Materialien, welche die Teilnehmer*innen des Seminars zur Mitarbeit in der von uns gestalteten Sitzung mit dem Titel ``Freier Wille als Illusion oder Notwendigkeit'' benötigen.

Die wichtigsten Ergebnisse etwaiger Diskussionen im Rahmen besagter Sitzung werden ebenfalls in diesem Webbook dokumentiert.

Bei Fragen zu den Inhalten dieses Webboks bzw. der Sitzung zögern Sie bitte nicht die Autor*innen zu kontaktieren.

\hypertarget{authors}{%
\section*{Die Autor*innen}\label{authors}}
\addcontentsline{toc}{section}{Die Autor*innen}

\hypertarget{eerhard}{%
\subsection*{Emma Erhard}\label{eerhard}}
\addcontentsline{toc}{subsection}{Emma Erhard}

Pronomen: sie/ihr (dt.) bzw. she/her (engl.)\\
Studierender Mensch - Universität Konstanz\\
Studienfach: Psychologie (M.Sc.)\\

Mail: \href{mailto:emma.erhard@uni-konstanz.de?subject=Freier\%20Wille\%20als\%20Illusion\%20oder\%20Notwendigkeit}{emma.erhard@uni-konstanz.de}

\hypertarget{hcoueslati}{%
\subsection*{Hamilkar Constantin Oueslati}\label{hcoueslati}}
\addcontentsline{toc}{subsection}{Hamilkar Constantin Oueslati}

Pronomen: dey/deren/denen (dt.) bzw. they/their/them (engl.)\\
Studierender Mensch - Universität Konstanz\\
Studienfach: Psychologie (M.Sc.)\\

Mail: \href{mailto:hamilkar-constantin.oueslati@uni-konstanz.de?subject=Freier\%20Wille\%20als\%20Illusion\%20oder\%20Notwendigkeit}{hamilkar-constantin.oueslati@uni-konstanz.de}

Web: \url{https://hco-consulting.eu}

\hypertarget{nsteimel}{%
\subsection*{Naima Steimel}\label{nsteimel}}
\addcontentsline{toc}{subsection}{Naima Steimel}

Pronomen: sie/ihr (dt.) bzw. she/her (engl.)\\
Studierender Mensch - Universität Konstanz\\
Studienfach: Psychologie (M.Sc.)\\

Mail: \href{mailto:naima.steimel@uni-konstanz.de?subject=Freier\%20Wille\%20als\%20Illusion\%20oder\%20Notwendigkeit}{naima.steimel@uni-konstanz.de}

\hypertarget{lecturer}{%
\section*{Der Dozent des Seminars}\label{lecturer}}
\addcontentsline{toc}{section}{Der Dozent des Seminars}

\hypertarget{dbeis}{%
\subsection*{Dr.~Daniel Beis}\label{dbeis}}
\addcontentsline{toc}{subsection}{Dr.~Daniel Beis}

Pronomen: er/ihm (dt.) bzw. he/him (engl.)\\
Wissenschaftlicher Mitarbeiter am Institut für Ernährungswissenschaften\\
Justus-Liebig-Universität Gießen\\

Mail: \href{mailto:?subject=Seminar\%20Freier\%20Wille\%20als\%20Illusion\%20oder\%20Notwendigkeit\%20-\%20Universität\%20Konstanz}{daniel.beis@ernaehrung.uni-giessen.de}

\hypertarget{intro}{%
\chapter{Einführung}\label{intro}}

\hypertarget{freewill-ynmi}{%
\section{Freier Wille? - Ja, Nein, Vielleicht, Egal}\label{freewill-ynmi}}

Im Rahmen dieser Seminarsitzung möchten wir uns mit euch auf kreative Art und Weise mit folgende Fragestellungen auseinandersetzen:

\begin{quote}
\emph{(1) Existiert der freie Wille wirklich oder ist er schlicht eine Illusion?}
\end{quote}

\begin{quote}
\emph{(2) Kann unsere Gesellschaft nur dann funktionieren, wenn der freie Wille existiert?}
\end{quote}

\begin{quote}
\emph{(3) Falls der freie Wille nicht existieren sollte, müssen wir an dessen Existenz glauben, um das Funktionieren unserer Gesellschaft sicherstellen zu können?}
\end{quote}

Die kritische Betrachtung solch abstrakter Fragestellungen ist leider oft alles andere als leicht. Aus diesem Grund wollen wir mit euch die besagten Fragestellungen anhand eines deutlich greifbareren Konzeptes kritisch diskutieren: der \textbf{Schuldfähigkeit}.

\hypertarget{freewill-blame}{%
\section{Freier Wille und Schuldfähigkeit}\label{freewill-blame}}

Bevor wir uns an der Beantwortung der obigen Fragen versuchen, lasst uns zuerst eine Antwort auf \textbf{die Frage} finden, \textbf{die immer die erste sein sollte}:

\begin{quote}
\emph{Wieso ist die Antwort auf diese Frage(n) von Relevanz?}
\end{quote}

Zur Beantwortung der \textbf{ersten Frage} lasst uns einige \textbf{potentielle} Implikationen von zwei Antwortmöglichkeiten für Frage (1) im Hinblick auf die Schuldfähigkeit einer kriminellen Person betrachten.

\textbf{Der freie Wille existiert.}

\begin{itemize}
\item
  Personen sind generell dazu fähig, sich bewusst dafür oder dagegen zu entscheiden ein bestimmtes Verhalten zu zeigen.

  \begin{itemize}
  \tightlist
  \item
    Personen sind im rechtlichen Sinne generell als \textbf{schuldfähig} zu betrachten.
  \end{itemize}
\item
  Personen sind \textbf{verantwortlich} für das Verhalten, welches sie bewusst zeigen bzw. nicht zeigen.

  \begin{itemize}
  \tightlist
  \item
    Personen \textbf{tragen} die \textbf{Schuld} für ihre kriminellen Handlungen bzw. für das kriminelle Unterlassen bestimmter Handlungen.
  \end{itemize}
\end{itemize}

\textbf{Der freie Wille existiert nicht.}

\begin{itemize}
\item
  Ob Personen ein bestimmtes Verhalten zeigen oder nicht, hängt \textbf{nicht} von bewussten Entscheidungsprozessen ab.

  \begin{itemize}
  \tightlist
  \item
    Personen sind im rechtlichen Sinne generell als \textbf{nicht schuldfähi}g zu betrachten.
  \end{itemize}
\item
  Personen sind \textbf{nicht verantwortlich} für das Verhalten, welches sie zeigen bzw. nicht zeigen.

  \begin{itemize}
  \tightlist
  \item
    Personen \textbf{tragen keine Schuld} für ihre kriminellen Handlungen bzw. für das kriminelle Unterlassen bestimmter Handlungen.
  \end{itemize}
\end{itemize}

Eine mögliche Antwort auf die \textbf{erste Frage} lautet dementsprechend:

\begin{quote}
\emph{Die Antworten auf Fragen (1) bis (3) sind von Relevanz, da diese beispielsweise bedeutende Implikationen für die Rechtsprechung in unserer Gesellschaft haben können.}
\end{quote}

\hypertarget{freewill-atcourt}{%
\chapter{Der Freie Wille Vor Gericht}\label{freewill-atcourt}}

Die zentralen Fragestellungen wurden definiert. Wir haben uns für ein greifbares und relevantes Konzept entschieden, anhand dessen wir diese diskutieren möchten. Wir sind uns im Klaren über die Relevanz der besagter Fragestellungen.

Wie können wir nun die genannten Fragestellungen anhand des Konzepts der Schuldfähigkeit auf kreative Art und Weise kritisch diskutieren?

Die Antwort lautet. \emph{Mit einem Rollenspiel!}

\begin{quote}
\textbf{\emph{Die Richter*innen rufen die Anwesenden zur Ordnung. Die Verhandlung beginnt!}}
\end{quote}

\hypertarget{deathofanicon}{%
\section{Die Ermordung einer Ikone}\label{deathofanicon}}

Im Rahmen einer \textbf{fiktiven Gerichtsverhandlung} soll entschieden werden, ob der \textbf{Angeklagte Daniel B.} (unser Dozent) des \textbf{Mordes} an seinem Kollegen \textbf{Sigmund F.} schuldig ist.

\textbf{Die Beweislage scheint eindeutig.} Alle vorhandenen Beweisstücke, die gerichtsmedizinische Untersuchung sowie die überlieferte Zeugenaussage von Carl Gustav J. lassen auf nur einen möglichen Tathergang schließen:

\begin{quote}
\emph{Am Morgen des 16.10.2022 nutzte Daniel B. einen Vortex-Manipulator, um zurück in das Jahr 1907 zu reisen. Am späten Abend des 03.11.1907 verschaffte sich Daniel B. sodann Zugang zu der Wohnung des Opfers in der Berggasse 19 in Wien. Laut der Aussage des Zeugen Carl Gustav J. ``stürmte'' Daniel B. in das Studierzimmer des Opfers und unterbrach unter Verwendung ``äußerst grotesker Flüche'' das erste Treffen zwischen dem Opfer und dem Zeugen. Als sich das Opfer auch nach mehrmaligen Aufforderungen von Daniel B. weigerte seine ``irrsinnige'' Sexualtheorie aufzugeben, begann Daniel B. ``wie besessen'' mit einem schweizer Taschenmesser auf das Opfer Sigmund F. einzustechen. Wenig später erlag Sigmund F. seinen Verletzungen.}
\end{quote}

\hypertarget{blameless}{%
\section{Ein Schuldloser Mord?}\label{blameless}}

Ausgehend von den dem Gericht vorliegenden Beweismitteln und der Zeugenaussage von Car Gustav J. scheint es außer Frage zu stehen, dass Daniel B. Sigmund F. ermordet hat.

Dies bedeutet jedoch nicht, dass Daniel B. des Mordes an Sigmund F. auch schuldig gesprochen werden kann. Lediglich wenn das Gericht basierend auf den Ausführungen der Staatsanwaltschaft und der Verteidigung zu dem Schluss kommen, dass Daniel B. zum Zeitpunkt des Mordes schuldfähig war, kann ein Schuldspruch erfolgen.

Daher bitten die ehrenwerten Richter*innen E. Erhard, H. C. Oueslati und N. Steimel die Teilnehmer*innen dieses Seminars sich in zwei Gruppen (Staatsanwaltschaft und Verteidigung) aufzuteilen und die Anklage gegen bzw. die Verteidigung von Daniel B. vorzubereiten.

\hypertarget{prosecution}{%
\section{Die Staatsanwaltschaft}\label{prosecution}}

Die Staatsanwaltschaft besteht aus den Teilnehmer*innen des Seminars mit den folgenden Initialen: \texttt{A.\ C.}, \texttt{A.\ K.}, \texttt{A.\ W.}, \texttt{A.\ B.}, \texttt{A.\ G.}, \texttt{D.\ M.}, \texttt{D.\ K.}, \texttt{E.\ L.} und \texttt{J.\ H.}

Die Staatsanwaltschaft wird gebeten alle in dem Kapitel 3 \protect\hyperlink{prosecution-evidence}{\emph{Beweismittel der Staatsanwaltschaft}} zur Verfügung stehenden Materialien und Ressourcen zu nutzen, um die Anklage gegen Daniel B. vorzubereiten. Ihr obliegt es zweifelsfrei darzulegen, dass Daniel B. zum Zeitpunkt des Tat \textbf{schuldfähig} war.

Des Weiteren bitten die ehrenwerten Richter*innen die Staatsanwaltschaft ein \textbf{Plädoyer} (Schlusswort) vorzubereiten. Zu Dokumentationszwecken sollte besagtes Schlusswort (zumindest in Stichworten) \textbf{niedergeschrieben} werden.

\hypertarget{sachverstuxe4ndiger-immanuel-kant}{%
\subsection{Sachverständiger Immanuel Kant}\label{sachverstuxe4ndiger-immanuel-kant}}

Die Teilnehmer*in mit den Initialen \texttt{A.\ G.} wird im Rahmen der Verhandlung die Rolle des \textbf{Sachverständigen} \textbf{Immanuel Kant} übernehmen. In seiner*ihrer Rolle kann und soll \texttt{A.\ G.} sowohl von der Staatsanwaltschaft als auch von der Verteidigung bzgl. des Themenkomplexes ``Freier Wille'' befragt werden.

Der*die Teilnehmer*in \texttt{A.\ G.} wird gebeten zur Vorbereitung seiner*ihrer Aussage als \textbf{Immanuel Kant} die Materialien in Kapitel 3.8 \protect\hyperlink{pr-expert}{\emph{Sachverständiger der Staatsanwaltschaft}} zu nutzen.

\hypertarget{defence}{%
\section{Die Verteidigung}\label{defence}}

Die Verteidigung besteht aus den Teilnehmer*innen des Seminars mit den folgenden Initialen: \texttt{J.\ M.}, \texttt{K.\ G.}, \texttt{K.\ H.}, \texttt{L.\ H.}, \texttt{L.\ S.}, \texttt{L.\ G.}, \texttt{M.\ P.}, \texttt{M.\ L.}, \texttt{M.\ H.} und \texttt{V.\ B.}

Die Verteidigung wird hingegen gebeten alle in dem Kapitel 4 \protect\hyperlink{defence-evidence}{\emph{Beweismittel der Verteidigung}} zur Verfügung stehenden Materialien und Ressourcen zu nutzen, um die Verteidigung von Daniel B. vorzubereiten. Ihr obliegt es zweifelsfrei darzulegen, dass Daniel B. zum Zeitpunkt des Tat \textbf{nicht schuldfähig} war.

Des Weiteren bitten die ehrenwerten Richter*innen die Verteidigung ein \textbf{Plädoyer} (Schlusswort) vorzubereiten. Zu Dokumentationszwecken sollte besagtes Schlusswort (zumindest in Stichworten) \textbf{niedergeschrieben} werden.

\hypertarget{sachverstuxe4ndiger-benjamin-libet}{%
\subsection{Sachverständiger Benjamin Libet}\label{sachverstuxe4ndiger-benjamin-libet}}

Die Teilnehmer*in mit den Initialen \texttt{L.\ G.} wird im Rahmen der Verhandlung die Rolle des \textbf{Sachverständigen} \textbf{Benjamin Libet} übernehmen. In seiner*ihrer Rolle kann und soll \texttt{L.\ G.} sowohl von der Verteidigung als auch von der Staatsanwaltschaft bzgl. des Themenkomplexes ``Freier Wille'' befragt werden.

Der*die Teilnehmer*in \texttt{L.\ G.} wird gebeten zur Vorbereitung seiner*ihrer Aussage als \textbf{Benjamin Libet} die Materialien in Kapitel 4.11 \protect\hyperlink{def-expert}{\emph{Sachverständiger der Verteidigung}} zu nutzen.

\begin{longtable}[]{@{}
  >{\raggedright\arraybackslash}p{(\columnwidth - 4\tabcolsep) * \real{0.1194}}
  >{\raggedright\arraybackslash}p{(\columnwidth - 4\tabcolsep) * \real{0.4179}}
  >{\raggedright\arraybackslash}p{(\columnwidth - 4\tabcolsep) * \real{0.4627}}@{}}
\caption{Staatsanwaltschaft und Verteidigung im Überblick}\tabularnewline
\toprule()
\begin{minipage}[b]{\linewidth}\raggedright
\end{minipage} & \begin{minipage}[b]{\linewidth}\raggedright
Staatsanwaltschaft
\end{minipage} & \begin{minipage}[b]{\linewidth}\raggedright
Verteidigung
\end{minipage} \\
\midrule()
\endfirsthead
\toprule()
\begin{minipage}[b]{\linewidth}\raggedright
\end{minipage} & \begin{minipage}[b]{\linewidth}\raggedright
Staatsanwaltschaft
\end{minipage} & \begin{minipage}[b]{\linewidth}\raggedright
Verteidigung
\end{minipage} \\
\midrule()
\endhead
\textbf{Mitglieder*innen} & \texttt{A.\ C.}, \texttt{A.\ K.}, \texttt{A.\ W.}, \texttt{A.\ B.}, \texttt{A.\ G.}, \texttt{D.\ M.}, \texttt{D.\ K.}, \texttt{E.\ L.} und \texttt{J.\ H.} & \texttt{J.\ M.}, \texttt{K.\ G.}, \texttt{K.\ H.}, \texttt{L.\ H.}, \texttt{L.\ S.}, \texttt{L.\ G.}, \texttt{M.\ P.}, \texttt{M.\ L.}, \texttt{M.\ H.} und \texttt{V.\ B.} \\
\textbf{Sachverständige*r} & \texttt{A.\ G.} & \texttt{L.\ G.} \\
\bottomrule()
\end{longtable}

\hypertarget{timetable}{%
\section{Der Ablauf der Verhandlung}\label{timetable}}

\hypertarget{timetable-sa}{%
\subsection{Samstag, den 14.01.2023}\label{timetable-sa}}

\begin{longtable}[]{@{}
  >{\raggedright\arraybackslash}p{(\columnwidth - 4\tabcolsep) * \real{0.0573}}
  >{\raggedright\arraybackslash}p{(\columnwidth - 4\tabcolsep) * \real{0.2188}}
  >{\raggedright\arraybackslash}p{(\columnwidth - 4\tabcolsep) * \real{0.7240}}@{}}
\caption{Ablauf der Verhandlung am 14.01.2023}\tabularnewline
\toprule()
\begin{minipage}[b]{\linewidth}\raggedright
Uhrzeit
\end{minipage} & \begin{minipage}[b]{\linewidth}\raggedright
Titel des Elements
\end{minipage} & \begin{minipage}[b]{\linewidth}\raggedright
Beschreibung
\end{minipage} \\
\midrule()
\endfirsthead
\toprule()
\begin{minipage}[b]{\linewidth}\raggedright
Uhrzeit
\end{minipage} & \begin{minipage}[b]{\linewidth}\raggedright
Titel des Elements
\end{minipage} & \begin{minipage}[b]{\linewidth}\raggedright
Beschreibung
\end{minipage} \\
\midrule()
\endhead
16:30 Uhr & \textbf{Impulsreferat} & Einführung in das Thema und Erläuterung des Ablaufs der Sitzung \\
16:40 Uhr & \textbf{Vorbereitung der Verhandlung} & Sichtung und Diskussion der Materialien innerhalb der Gruppen \textbf{Staatsanwaltschaft} und \textbf{Verteidigung}. \\
17:25 Uhr & \textbf{Eröffnung der Gerichtsverhandlung} & Verlesung der Anklageschrift durch die Richter*innen und Beginn der Verhandlung. \\
17:30 Uhr & \textbf{Beweisanträge der Staatsanwaltschaft} & Darlegung der Argumente bzw. Beweismittel \textbf{für} die Schuldfähigkeit von Daniel B. (inkl. der Befragung des Sachverständigen \texttt{A.\ G.}). \\
17:45 Uhr & \textbf{Beweisanträge der Verteidigung} & Darlegung der Argumente bzw. Beweismittel \textbf{gegen} die Schuldfähigkeit von Daniel B. (inkl. der Befragung des Sachverständigen \texttt{L.\ G.}). \\
18:00 Uhr & \textbf{Ende des ersten Verhandlungstages} & \\
\bottomrule()
\end{longtable}

\hypertarget{timetable-sun}{%
\subsection{Sonntag, den 15.01.2023}\label{timetable-sun}}

\begin{longtable}[]{@{}
  >{\raggedright\arraybackslash}p{(\columnwidth - 4\tabcolsep) * \real{0.0902}}
  >{\raggedright\arraybackslash}p{(\columnwidth - 4\tabcolsep) * \real{0.3033}}
  >{\raggedright\arraybackslash}p{(\columnwidth - 4\tabcolsep) * \real{0.6066}}@{}}
\caption{Ablauf der Verhandlung am 15.01.2023}\tabularnewline
\toprule()
\begin{minipage}[b]{\linewidth}\raggedright
Uhrzeit
\end{minipage} & \begin{minipage}[b]{\linewidth}\raggedright
Titel des Elements
\end{minipage} & \begin{minipage}[b]{\linewidth}\raggedright
Beschreibung
\end{minipage} \\
\midrule()
\endfirsthead
\toprule()
\begin{minipage}[b]{\linewidth}\raggedright
Uhrzeit
\end{minipage} & \begin{minipage}[b]{\linewidth}\raggedright
Titel des Elements
\end{minipage} & \begin{minipage}[b]{\linewidth}\raggedright
Beschreibung
\end{minipage} \\
\midrule()
\endhead
9:30 Uhr & \textbf{Plädoyer der Staatsanwaltschaft} & Schlusswort der Staatsanwaltschaft. \\
9:32 Uhr & \textbf{Plädoyer der Verteidigung} & Schlusswort der Verteidigung. \\
09:35 Uhr & \textbf{Urteilsverkündung} & Verkündung und Begründung des Urteils durch die Richter*innen. \\
09:45 Uhr & \textbf{Reflexion der Sitzung} & Kritische Reflexion der Sitzung im Rahmen einer freien Gruppendiskussion \\
10:00 Uhr & \textbf{Ende der Verhandlung} & \\
\bottomrule()
\end{longtable}

\hypertarget{prosecution-evidence}{%
\chapter{Beweismittel der Staatsanwaltschaft}\label{prosecution-evidence}}

Eure Position als Staatsanwaltschaft ist die folgende:

\begin{quote}
\emph{Daniel B. ist als zum Zeitpunkt der Tat \textbf{schuldfähig} anzusehen. Daher muss er \textbf{schuldig} gesprochen werden.}
\end{quote}

Eure Position basiert auf einer oder beider der folgenden Annahmen:

\begin{itemize}
\item
  Daniel B. besitzt einen \textbf{freien Willen} und hat sich \textbf{bewusst} dazu entschieden Sigmund F. zu ermorden.
\item
  Daniel B. besitzt \textbf{keinen} \textbf{freien Willen} und konnte sich daher \textbf{nicht} \textbf{bewusst} dazu entscheiden Sigmund F. zu ermorden. \textbf{Nichtsdestotrotz} muss Daniel B. \textbf{schuldig} gesprochen werden, da unsere gesamte Gesellschaft die \textbf{Illusion} teilt, dass der \textbf{freie Wille} existiert.
\end{itemize}

Nutzt bitte die folgenden Materialien, um eine sinnvolle Argumentation aufzubauen. Selbstverständlich könnt ihr zur Vorbereitung weitere seriöse Quellen eurer Wahl nutzen.

\hypertarget{pr-ev1}{%
\section{\#1 Haben wir einen freien Willen?}\label{pr-ev1}}

Auszug aus dem Aufsatz ``Haben wir einen freien Willen?'' von Benjamin Libet \citeyearpar{Libet2004}

\begin{quote}
``Ich habe mich dieser Frage auf experimentelle Weise genähert. Freien Willenshandlungen geht eine spezifische elektrische Veränderung im Gehirn voraus (das ›Bereitschaftspotentiak‹, BP), das 550 ms vor der Handlung einsetzt. Menschliche Versuchspersonen wurden sich der Handlungsintention 350-400 ms nach Beginn von BP bewußt, aber 200 ms vor der motorischen Handlung. Der Willensprozeß wird daher unbewußt eingeleitet. Aber die Bewußtseinsfunktion kann den Ausgang immer noch steuern; sie kann die Handlung durch ein Veto verbieten. Willensfreiheit ist daher nicht ausgeschlossen. Diese Befunde stellen Beschränkungen für mögliche Ansichten darüber dar, wie der freie Wille funktionieren könnte; er würde eine Willenshandlung nicht einleiten, würde aber den Vollzug der Handlung steuern.'' (S.268)
\end{quote}

\hypertarget{pr-ev2}{%
\section{\#2 Verschaltungen legen uns fest}\label{pr-ev2}}

Auszüge aus dem Aufsatz ``Verschaltungen legen uns fest: Wir sollten aufhören, von Freiheit zu sprechen'' von Wolf Singer \citeyearpar{Singer2004}

\begin{quote}
``Wir sprechen von freiem Willen und wissen, was wir darunter zu verstehen haben. Wir begreifen uns als Wesen, die über Intentionalität verfügen, die fähig sind zu entscheiden, initiativ zu werden und zielbewußt in den Ablauf der Welt einzugreifen. Wir erfahren uns als freie und folglich als verantwortende, autonome Agenten. Es scheint uns, als gingen unsere Entscheidungen unseren Handlungen voraus und wirkten auf Prozesse im Gehirn ein, deren Konsequenz dann die Handlung ist. Diese Überzeugungen erwachsen aus der Erfahrung, daß wir uns unserer eigenen Empfindungen, Wahrnehmungen, Erinnerungen, Absichten und Handlungen gewahr sein und auf diese Einfluß nehmen können.'' (S. 33)
\end{quote}

\begin{quote}
``Diese Sicht hat Konsequenzen für die Beurteilung von Fehlverhalten. Ein Beispiel: Eine Person begeht eine Tat, offenbar bei klarem Bewußtsein, und wird für voll verantwortlich erklärt. Zufällig entdeckt man aber einen Tumor in Strukturen des Frontalhirns, die benötigt werden, um erlernte soziale Regeln abzurufen und für Entscheidungsprozesse verfügbar zu machen. Der Person würde Nachsicht zuteil. Der gleiche »Defekt« kann aber auch unsichtbare neuronale Ursachen haben. Genetische Dispositionen können Verschaltungen hervorgebracht haben, die das Speichern oder Abrufen sozialer Regeln erschweren, oder die sozialen Regeln wurden nicht rechtzeitig und tief genug eingeprägt, oder es wurden von der Norm abweichende Regeln erlernt, oder die Fähigkeit zur rationalen Abwägung wurde wegen fehlgeleiteter Prägung ungenügend ausdifferenziert. Diese Liste ließe sich nahezu beliebig verlängern. Keiner kann anders, als er ist.

Diese Einsicht könnte zu einer humaneren, weniger diskriminierenden Beurteilung von Mitmenschen führen, die das Pech hatten, mit einem Organ volljährig geworden zu sein, dessen funktionelle Architektur ihnen kein angepaßtes Verhalten erlaubt. Menschen mit problematischen Verhaltensdispositionen als schlecht oder böse abzuurteilen bedeutet nichts anderes, als das Ergebnis einer schicksalhaften Entwicklung des Organs, das unser Wesen ausmacht, zu bewerten. Überschreitet das Fehlverhalten eine Toleranzgrenze, drohen wir mit Sanktionen. Interessanterweise fallen diese Maßnahmen um so drastischer aus, je mehr wir davon ausgehen können, daß dem Delinquenten die Variablen, auf denen die Entscheidung basierte, bewußt sein müßten. Offenbar ahnden wir Verstöße dann besonders streng, wenn sie gegen explizit Gewußtes begangen werden, gegen Wertordnungen also, die über Erziehungsprozesse im deklarativen Gedächtnis verankert wurden. Wir begründen dies, indem wir bewußten Entscheidungen ein besonderes Maß an Freiheit zuschreiben und daraus besondere Schuldfähigkeit, Verantwortlichkeit und Sanktionsnotwendigkeit ableiten.

An dieser Praxis würde die differenziertere Sicht der Entscheidungsprozesse, zu der neurobiologische Erkenntnisse zwingen, wenig ändern. Die Gesellschaft darf nicht davon ablassen, Verhalten zu bewerten. Sie muß natürlich weiterhin versuchen, durch Erziehung, Belohnung und Sanktionen Entscheidungsprozesse so zu beeinflussen, daß unerwünschte Entscheidungen unwahrscheinlicher werden, sie muß Delinquenten die Chance einräumen, durch Lernen zu angepaßteren Entscheidungen zu finden, und - wenn all dies erfolglos bleibt - sich durch Freiheitsentzug schützen. Nur die Argumentationslinie wäre eine andere. Sie trüge den hirnphysiologischen Erkenntnissen Rechnung, ersetzte die konfliktträchtige Zuschreibung graduierter »Freiheit« und Verantwortlichkeit durch bewußte und unbewußte Prozesse und eröffnete damit einen vorurteilsloseren Raum zur Beurteilung und Bewertung von »normalem« und »abweichendem« Verhalten. Die schwer nachvollziehbare Dichotomie einer Person in freie und unfreie Komponenten wäre damit überwunden. Die Person als ganze würde nach wie vor für all das zur Rechenschaft gezogen, was sie fühlt, denkt und tut, und diese Beurteilung umfaßte unbewußte und bewußte Faktoren gleichermaßen. Diese Sichtweise trüge der trivialen Erkenntnis Rechnung, daß eine Person tat, was sie tat, weil sie im fraglichen Augenblick nicht anders konnte - denn sonst hätte sie anders gehandelt. Da im Einzelfall nie ein vollständiger Überblick über die Determinanten einer Entscheidung zu gewinnen ist, wird sich die Rechtsprechung nach wie vor an pragmatischen Regelwerken orientieren. Es könnte sich aber lohnen, die geltende Praxis im Lichte der Erkenntnisse der Hirnforschung einer Überprüfung auf Kohärenz zu unterziehen.'' (S. 63-64)
\end{quote}

\hypertarget{pr-ev3}{%
\section{\#3 Gründe zählen}\label{pr-ev3}}

Auszüge aus dem Aufsatz ``Gründe zählen. Über einige Schwierigkeiten des Bionaturalismus'' von Lutz Wingert \citeyearpar{Lutz2004}

\begin{quote}
``»Freiheit« ist jene kleine Bewegung, die aus einem völlig {[}\ldots{]} bedingten Wesen einen Menschen macht, der nicht in allem das darstellt, was von seinem Bedingtsein herrührt. So wird aus Jean Genet ein Dichter, obwohl er ganz dazu bedingt war, ein Dieb zu werden.«

Jean-Paul Sartre, von dem diese Sätze stammen, meinte mit dem »Bedingtsein« soziale Faktoren, beispielsweise - wie Genet - ein Findelkind zu sein, das unter der Vormundschaft einer seelenlosen Sozialbürokratie steht, bettelarm zu sein und von klein auf mit sozialer Verachtung gestraft zu werden. Sartre dachte nicht an biologisch beschreibbare Faktoren, die einen Menschen in seinem Verhalten so festlegen, daß kein Raum für Freiheit und Verantwortlichkeit besteht. Dazu war er zu naturvergessen. Aber auch mit mehr Sinn und Kenntnis der (Neuro-)Biologie hätte er wohl seine common-sense-konforme These aufrechterhalten: Gewöhnlich haben - erwachsene und gesunde - Menschen einen Handlungsspielraum, der Platz für Freiheit und Verantwortlichkeit läßt.'' (S. 194)
\end{quote}

\begin{quote}
``Denn Handlungen - das ist die obere Pramisse - sind Körperbewegungen, die von bewußten, spontanen Entschlüssen verrsacht werden.

Die deterministische Schlußfolgerung wird von den experimentellen Befunden aber nicht gedeckt. Eine Entscheidung ist kein bloßer Ruck. Das ist zu Recht immer wieder bemerkt worden. Die Entscheidung ist auch ein Wählen und Zurückweisen. Wenn man sich für eine Handlung entscheidet, dann weist man deren Unterlassung zurück. Unterlassung ist aber mehr als blockierte Dopaminausschüttung. Zum Handeln gehört ein Urteil: » Es ist (jetzt) besser, das und das zu tun, als es zu unterlassen.« Die Probanden von Libet \& Co.~hatten allerdings keine Hinsicht, unter der es ihnen besser erschien, diese oder jene Taste zu drücken. Deshalb mußten sie im Labor Handlungen simulieren. Ihre nichtsimulierte Handlung war, den Aufforderungen des Laborleiters Folge zu leisten. Und dazu hatten sie sich lange vor ihrem Tastendrücken entschieden. So betrachtet gleichen die im Labor ausgeführten willkürmotorischen Bewegungen dem Sprung des Tormanns beim Elfmeter, der schon vor dem bewußten Registrieren des Torschusses zu reagieren beginnt. Sie sind nicht weniger freiwillig als die habitualisierte Parade eines geübten und warmgeschossenen Torwarts.

Die Entscheidung der Probanden, den Aufforderungen der Versuchsleiter zu folgen, fiel gewiß nicht vom Himmel. Aber daß dieses Faktum eine deterministische Schlußfolgerung stützt, setzt eine unplausible Freiheitsauffassung voraus. Man könnte sie das Würfelwurfmodell der Freiheit nennen: Frei bin ich, wenn ich auch anders handeln könnte - gesetzt den Fall, ich wollte nur anders. Das frei Gewollte ist wie ein Würfelwurf. Der Wurf könnte so ausfallen - ich könnte das wollen. Der Wurf konnte aber auch so ausfallen - ich konnte auch jenes wollen. Und je nachdem, wie die Würfel fallen, je nachdem, was mein Wille zufälligerweise gerade ist, so handele ich eben - wenn ich frei bin. Der freie Mensch hat keinen festlegenden Grund, zu wollen, was er will. Dieses Würfelwurfmodell der Freiheit führt dazu, den freien Willen als ein unbeschriebenes Blatt zu denken. Kein Wunder, daß man ein solches Blatt nicht findet.

Befreit man sich von diesem Bild der Freiheit, dann kann man Freiheit als Fähigkeit zur Selbstbindung im Handeln durch Gründe verstehen. Frei ist man in seinem Tun, wenn man auch anders handeln könnte, gesetzt den Fall, man hätte einen Grund dafür, anders zu handeln.'' (S. 197-198)
\end{quote}

\begin{quote}
``Ich mag mich über den rücksichtslos rempelnden Passanten auf der Bahnhofstreppe ärgern. Bisweilen jedoch verraucht mein Ärger und verstummt mein Lospoltern, wenn ich erkenne, daß er extrem kurzsichtig ist. Ein Grund (»Er kann nichts dafür«) ändert mein Handeln. Gründe sind nicht immer bloß sozialverträgliche Rationalisierungen, die keine Rolle im Handeln spielen. Sie zählen. Und sie erklären in vielen Fällen am besten den Erfolg und Mißerfolg unserer Handlungen.'' (S. 199)
\end{quote}

\begin{quote}
``Die organische Natur begrenzt und ermöglicht Freiheit - die soziale Welt übrigens auch. Die kognitiven Prozesse und Leistungen sind in dem Sinn von neuronalen Prozessen abhängig, daß sie nicht ohne diese möglich sind. Aber daß etwas nicht ohne etwas anderes vorkommen kann, bedeutet nicht, das es damit zusammenfällt. Ebensowenig müssen die Eigenschaften eines lebenden Systems durch die Eigenschaften seiner Teile festgelegt sein.'' (S. 202)
\end{quote}

\begin{quote}
``Man wird hier keinen schritt weiterkommen, wenn man sich nicht von einem tiefsitzenden Vorurteil befreit. Dem Vorurteil nämlich, daß nur das, was mit den Erkenntnismitteln der Biologie, der medizinischen Psychologie, Chemie und Physik erfaßt werden kann, real ist. Damit würde man die Grenzen der naturwissenschaftlichen Erkenntnismittel mit unseren Erkenntnisgrenzen gleichsetzen.'' (S. 204)
\end{quote}

\hypertarget{pr-ev4}{%
\section{\#4 Warum noch debattieren?}\label{pr-ev4}}

Auszug aus dem Aufsatz ``Warum noch debattieren? Determinismus als Diskurskiller'' von Gerhard Kaiser \citeyearpar{Kaiser2004}

\begin{quote}
``Aber derselbe Wolf Singer, der von Komplementarität spricht, verwendet fortgesetzt grenzüberschreitend das Kausalitätsschema (S. 29), nennt Gedanken »Folge neuronaler Prozesse« (S. 15) und weitet die deterministische Position sogar dahin aus, daß »auch die kulturelle Umwelt determiniert« (S. 23). Die strikte Determination der geistigen Zustände und Aktivitäten des Menschen letztendlich durch die neuronalen Gegebenheiten gewinnt so die Qualität eines Glaubenssatzes.

Es ist ein Glaube, der weit über alles hinausgreift, was sich experimentell nachweisen läßt. Wie Singer wiederholt feststellt, gibt es enorme qualitative Sprünge von der immer noch relativ einfachen neuronalen Bearbeitung von Sinneseindrücken bis hin etwa zum Verstehen von Sprache (als nicht nur Sinneseindruck, sondern Mitteilung). Ahnlich weit ist der Sprung von der Verknüpfung neuronaler Reizzustände mit einfachen Handlungsentscheidungen etwa für die Öffnung einer rechten oder linken Tür, wie sie auch Primaten treffen, bis hin zur neuronalen Entsprechung für Wallensteins Erwägung: »Wär's möglich? Könnt' ich nicht mehr, wie ich wollte?« (»Wallensteins Tode«, IV, 4) oder zu der Überlegung: Determinieren neuronale Erregungszustände mein Denken und Handeln? Abgesehen davon wäre auch bei relativ einfachen experimentellen Ergebnissen in einem derartig sensiblen Bereich wie der Hirnforschung das Grundprinzip der Elementarteilchen-Physik zu bedenken: daß nämlich der beobachtete Vorgang durch den Vorgang der Beobachtung beeinflußt wird, daß dem Experiment also die Beobachterposition eingeschrieben ist.

Und nicht nur das. Naturwissenschaftliche Experimente gliedern ihr Beobachtungsfeld ab ovo aus der Lebenswelt des Beobachters aus, reduzieren ihren Gegenstand auf das Wiederholbare und Quantifizierbare und versachlichen den Experimentator selbst zur neutralen Beobachterinstanz.'' (S. 263-264)
\end{quote}

\hypertarget{pr-ev5}{%
\section{\#5 Infektion des Geistes}\label{pr-ev5}}

Auszüge aus dem Aufsatz ``Infektion des Geistes. Über philosophische Kategorienfehler'' von Gerd Kempermann \citeyearpar{Kempermann2004}

\begin{quote}
``Ohne Gehirn kein menschlicher Geist. Es ist ebenfalls eine Alltagserfahrung, daß der Wille so frei nicht ist. Die Psychologie der Politik und des Gesundheitsverhaltens sind Beispiele, wie Menschen anders handeln, als sie könnten und »eigentlich« wollen.'' (S. 235)
\end{quote}

\begin{quote}
``Der Mensch sei (je nach Ideologie) zu 10, 50 oder 90 Prozent durch die Umwelt und der Rest auf hundert Prozent durch seine Gene bestimmt. Das ist Unsinn. Der Mensch ist ganz durch seine Gene und ganz durch seine Umwelt bestimmt. Diese Wechselwirkung ist wörtlich zu nehmen.'' (S. 235)
\end{quote}

\begin{quote}
``Der freie Wille gehört wie die Menschenwürde in die Karegorie der Konstrukte, die Zuschreibungen sind. Freier Wille bleibt schon in der Eigenwahrnehmung unscharf. Man mag prinzipiell viel auf die eigene Autonomie halten, um sie doch in schwierigen Situationen, nicht nur in Strafverfahren, bereitwillig über Bord zu werfen und auf biologische Entlastungsgründe zurückzugreifen.'' (S. 236)
\end{quote}

\hypertarget{pr-ev6}{%
\section{\#6 Is Free Will an Illusion?}\label{pr-ev6}}

Ein TEDx Talk von \textbf{Nick Jankovic}, in dem er kritisch die Idee diskutiert, dass der freie Wille nichts weiter als eine Illusion ist \citep{Jankovic2021}.

\href{https://www.youtube.com/watch?v=J21bSGGvnD4}{Video: ``Is Free Will an Illusion?''}

\hypertarget{pr-ev7}{%
\section{\#7 Freier Wille: Was sagt die Wissenschaft?}\label{pr-ev7}}

Ein Video von \textbf{Ralph Caspers}, in dem er versucht den Kenntnisstand der Wissenschaft bzgl. der Frage nach der Existenz des freien Willens zusammenzufassen \citep{Caspers2021}.

\href{https://www.youtube.com/watch?v=45Iut50Cm_Q}{Video: ``Freier Wille: Was sagt die Wissenschaft?''}

\hypertarget{pr-expert}{%
\section{Sachverständiger der Staatsanwaltschaft}\label{pr-expert}}

In der Rolle des Sachverständigen \textbf{Immanuel Kant} vertritt \texttt{A.\ G.} die folgende Position:

\begin{quote}
\emph{Menschen sollten \textbf{trotz} der Auffassung des \textbf{Determinismus} für ihr Handeln \textbf{verantwortlich} gemacht werden.}
\end{quote}

Nutze bitte die folgenden Materialien zur Vorbereitung deiner Aussage.

\hypertarget{pr-expert-ev1}{%
\subsection{\#1 Der entlarvte Ruck}\label{pr-expert-ev1}}

Auszüge aus dem Aufsatz „Der entlarvte Ruck. Was sagt Kant den Gehirnforschern‟ von Otfried Höffe \citeyearpar{Höffe2004}

\begin{quote}
``Kant gibt dem Kritiker der Freiheit insofern recht, als jedes Ereignis, einschließlich jeder Handlung, sich auf Ursachen hinterfragen läßt. Man kann weder die Ursachenfrage von sich weisen noch deren Nichtbeantwortbarkeit belegen. Folglich ist jedes Ereignis potentiell determiniert. Wer aus diesem methodischen Determinismus aber jenen dogmatischen Determinismus ableitet, der die Freiheit für unmöglich erklärt, unterschlägt eine methodische Einschränkung: Ereignisse sind nur soweit determiniert, wie »man sich im Umkreis möglicher Erfahrung bewegt«.'' (S. 179)
\end{quote}

\begin{quote}
``Die für Kant entscheidende Willensfreiheit, die vielzitierte Autonomie des Willens, besteht nicht in irgendeiner Selbstbestimmung, sondern einer Selbstbestimmung dieser höchsten, dritten Stufe. Kant versteht unter dem Willen die Fähigkeit, sein Handeln an der Vorstellung gewisser Gesetze auszurichten. Frei ist dieser Wille, sofern er sich das Gesetz (Nomos) selbst gibt. Da »selbst‹ auf griechisch »autor« heißt, spricht Kant von Auto-nomie. Schon auf den niederen Freiheitsstufen folgt man einem Gesetz, das aber nicht aus dem Willen selbst stammt, sondern von woanders herkommt, weshalb Hetero-nomie vorliegt. Die Frage der Willensfreiheit entscheidet sich jedenfalls nicht - wie im Libet-Experiment - an einer »atomaren« Handlung, sondern an der Art des zugrundeliegenden Gesetzes. Da sich aber weder das Libet-Experiment noch dessen Verbesserungen, vielleicht auch Korrekturen mit ihm befassen, ist ihnen - so eine weitere Zwischenbilanz - rein thematisch die Willensfreiheit entzogen.

Ohne jeden moralisierenden Unterton bestimmt sich die Moral nach Kant als die nicht mehr steigerbare, als die absolute Höchstform des Guten. Erst ihr entspricht der philosophische Begriff von Willensfreiheit. Nicht in einem Willensruck besteht sie, sondern in dem Umstand, daß der Wille keinem fremden, sondern dem eigenen Gesetz folgt.

Daß dem Menschen diese Möglichkeit tatsächlich offensteht, erläutert Kant an einem überzeugenden Gedankenexperiment: Angenommen, jemand wird »unter Androhung der unverzögerten Todesstrafe« aufgefordert, »ein falsches Zeugnis wider einen ehrlichen Mann \ldots{} abzulegen, ob er da, so groß auch seine Liebe zum Leben sein mag, sie wohl zu überwinden für möglich halte. Ob er es tun würde, oder nicht, wird er vielleicht sich nicht getrauen zu versichern, daß es ihm aber möglich sei, muß er ohne Bedenken einräumen.«

Daß man um eines Vorteiles, vor allem des Überlebens willen einmal unmoralisch handelt und beispielsweise lügt, wird jeder für möglich halten. Ebenso hält er es aber für möglich, die Lüge zu verweigern. Daher fährt Kant fort: »Er urteilt also, daß er etwas kann, darum, weil er sich bewußt ist, daß er es soll, und erkennt in sich die Freiheit, die ihm sonst ohne das moralische Gesetz unbekannt geblieben wäre.«

Ob die Hirnforschung für diese Art Freiheit, die wirkliche Willensfreiheit, ein experimentum crucis, eine entscheidende Frage formulieren kann, weiß ich nicht. Mit der dem Menschen angeborenen Wißbegier bleibe ich neugierig, freilich auch skeptisch. Denn wie will man die Welt des Sollens mit Einsichten aus der Welt des Seins aus den Angeln heben? Solange man lediglich die Libet-Haggar-Eimer-Experimente anführt, dürfen wir jedenfalls Kant folgen und die Moral samt Willensfreiheit zumindest in einer ersten Stufe für real halten. Und wo man sich durch Erziehung und Selbsterziehung die Haltung der Ehrlichkeit erwirbt, insofern sich die volle Wirklichkeit der Moral zu eigen macht, deshalb selbst in schwieriger Lage ehrlich bleibt und ebenso hilfsbereit oder couragiert, dort zeigen Moral und Willensfreiheit ihre Realität.'' (S. 181-182)
\end{quote}

\hypertarget{pr-expert-ev2}{%
\subsection{\#2 Grundlegung zur Metaphysik der Sitten}\label{pr-expert-ev2}}

Auszüge aus dem Buch ``Grundlegung zur Metaphysik der Sitten'' von Immanuel Kant \citeyearpar{Kant1785}.

\begin{quote}
``{[}\ldots{]} wenn wir uns als frei denken, so versetzen wir uns als Glieder in die Verstandeswelt, und erkennen die Autonomie des Willens, samt ihrer Folge, der Moralität; denken wir uns aber als verpflichtet, so betrachten wir uns als zur Sinnenwelt und doch zugleich zur Verstandeswelt gehörig.''
\end{quote}

\begin{quote}
``Ich sage nun: ein jedes Wesen, das nicht anders als unter der Idee der Freiheit handeln kann, ist eben darum, in praktischer Rücksicht, wirklich frei. Nun behaupte ich: dass wir jedem vernünftigen Wesen, das einen Willen hat, notwendig auch die Idee der Freiheit leihen müssen, unter der es allein handle.''
\end{quote}

\begin{quote}
``Als ein vernünftiges, mithin zur intelligiblen Welt gehöriges Wesen, kann der Mensch die Kausalität seines eigenen Willens niemals anders, als unter der Idee der Freiheit denken; denn Unabhängigkeit von den bestimmten Ursachen der Sinnenwelt (dergleichen die Vernunft jederzeit sich selbst beilegen muss) ist Freiheit.''
\end{quote}

\hypertarget{pr-expert-ev3}{%
\subsection{\#3 Willensfreiheit und Determinismus bei Kant erklärt!}\label{pr-expert-ev3}}

Ein Video von \textbf{Samuel Jalalian}, in dem erklärt wird, wie wie Immanuel Kant zum Thema Determinismus und Willensfreiheit steht \citep{Jalalian2022}.

\href{https://www.youtube.com/watch?v=Z8zJVcFVvX4}{Video: ``Willensfreiheit und Determinismus bei Kant erklärt!''}

\hypertarget{defence-evidence}{%
\chapter{Beweismittel der Verteidigung}\label{defence-evidence}}

Eure Position als Verteidigung ist die folgende:

\begin{quote}
\emph{Daniel B. ist als zum Zeitpunkt der Tat \textbf{nicht schuldfähig} anzusehen. Daher muss er \textbf{frei} gesprochen werden.}
\end{quote}

Eure Position basiert auf der Annahme, dass Daniel B. \textbf{keinen} \textbf{freien Willen} besitzt und sich daher \textbf{nicht} \textbf{bewusst} dazu entscheiden konnte Sigmund F. zu ermorden.

Nutzt bitte die folgenden Materialien, um eine sinnvolle Argumentation aufzubauen. Selbstverständlich könnt ihr zur Vorbereitung weitere seriöse Quellen eurer Wahl nutzen.

\hypertarget{def-ev1}{%
\section{\#1 Haben wir einen freien Willen?}\label{def-ev1}}

Auszug aus dem Aufsatz ``Haben wir einen freien Willen?'' von Benjamin Libet \citeyearpar{Libet2004}

\begin{quote}
``Ich habe mich dieser Frage auf experimentelle Weise genähert. Freien Willenshandlungen geht eine spezifische elektrische Veränderung im Gehirn voraus (das ›Bereitschaftspotentiak‹, BP), das 550 ms vor der Handlung einsetzt. Menschliche Versuchspersonen wurden sich der Handlungsintention 350-400 ms nach Beginn von BP bewußt, aber 200 ms vor der motorischen Handlung.'' (S.268)
\end{quote}

\hypertarget{def-ev2}{%
\section{\#2 Verschaltungen legen uns fest}\label{def-ev2}}

Auszüge aus dem Aufsatz ``Verschaltungen legen uns fest: Wir sollten aufhören, von Freiheit zu sprechen'' von Wolf Singer \citeyearpar{Singer2004}

\begin{quote}
``Da wir, was tierische Gehirne betrifft, keinen Anlaß haben zu bezweifeln, daß alles Verhalten auf Hirnfunktionen beruht und somit den deterministischen Gesetzen physiko-chemischer Prozesse unterworfen ist, muß die Behauptung der materiellen Bedingtheiten von Verhalten auch auf den Menschen zutreffen.'' (S. 37)
\end{quote}

\begin{quote}
``Im folgenden soll der Versuch gemacht werden, die Bedingungen zu identifizieren, die es uns ermöglichen, uns als selbstbestimmende, frei entscheidende Wesen zu erfahren. Eine zentrale Rolle scheint hierbei dem Faktum zuzukommen, daß uns bei weitem nicht alle Vorgänge in unserem Gehirn bewußt werden. Vieles spricht dafür, daß nur die neuronalen Erregungsmuster zu bewußten Empfindungen und Wahrnehmungen führen, die in der Hirnrinde generiert werden. Von diesen wiederum dürfte jeweils nur ein Bruchteil ins Bewußtsein gelangen. Noch wissen wir wenig darüber, durch welche Eigenschaften sich die bewußtseinsfähigen von den unbewußt bleibenden Erregungszuständen unterscheiden. Manches spricht dafür, daß Erregungsmuster nur dann bewußt werden können, wenn ihnen »Aufmerksamkeit« geschenkt wird und sie dadurch ein kritisches Maß an Kohärenz, an Ordnung, an Synchronisation erlangen und diesen Zustand über hinreichend lange Zeit aufrechterhalten können.'' (S. 46)
\end{quote}

\begin{quote}
``Es scheint, als könnten Erregungsmuster erst dann bewußt werden, wenn sie ein gewisses Maß an Konsistenz erreicht haben, also als Ergebnis eines Verarbeitungsprozesses gewertet werden. Manche der vom Gehirn ausgewerteten Signale haben jedoch prinzipiell keinen Zugang zum Bewußtsein. Wir haben zum Beispiel keinen bewußten Zugriff zu Informationen über unseren Blutdruck oder das Niveau des Blutzuckerspiegels, obgleich diese Variablen sehr sorgfältig gemessen, vom Gehirn ausgewertet und in Regulationsprozesse umgesetzt werden.'' (S. 47)
\end{quote}

\begin{quote}
``Eine erste und vermutlich entscheidende Erfahrung mit der Zuschreibung von Autonomie und Freiheit machen wir schon als Kleinkinder. Eltern bedeuten den Kleinen fortwährend, sie sollten dies tun und jenes lassen, weil andernfalls diese oder jene Konsequenzen einträten. Diese Verweise und die mit ihnen verbundenen Sanktionen erzwingen den Schluß, man könne auch anders und müsse nur wollen. Wir erfahren also schon sehr früh eine Behandlung, die sich durch die Annahme rechtfertigt, wir seien frei in unseren Entscheidungen - eine Annahme, die sich über Erziehung verläßlich von Generation zu Generation tradiert. Wir machen uns also vermutlich eine im Laufe unserer Kulturgeschichte entwickelte Zuschreibung zu eigen, internalisieren sie und verfahren nach ihr. Möglich ist dies, weil wir bislang auf keine direkt erfahrbaren Widersprüche gestoßen sind. Wenn die Prämisse gilt, daß neuronale Prozesse erst dann bewußt werden können, wenn sie sich Lösungen nähern, dann bleibt die Erfahrung, frei zu sein, widerspruchsfrei, weil wir uns der Aktivitäten nicht gewahr werden, welche die Entscheidungen vorbereiten und zu anderen Lösungen hätten führen können.'' (S. 49)
\end{quote}

\begin{quote}
``Die meisten der Strebungen und Motive, die uns letztlich dazu gebracht haben, etwas Bestimmtes und nicht anderes zu tun, bleiben uns verborgen. Wir nehmen oft nur das Ergebnis solcher hirninterner Abwägungsprozesse wahr, schreiben uns dies dann im Moment der Bewußtwerdung als Ergebnis unserer »freien« Entscheidung zu, können es dann noch mit anderen, ebenfalls bewußten Argumenten abwägen und gegebenenfalls modifizieren und erfahren uns so als Herr über unsere Entscheidungen. Da wir unbewußte Motive per definitionem nicht wahrnehmen, ergibt sich kein erfahrbarer Widerspruch zwischen der grundsätzlichen Bedingtheit unserer Entscheidungen und unserem Eindruck, wir träfen sie frei. Weil uns alle vorbereitenden, »vorbewußten« Vorgänge in unserem Gehirn verborgen bleiben, erscheint uns das, was im Bewußtsein aufscheint, als nicht verursacht. Nun lehren uns aber alle Erfahrungen, daß nichts ohne Ursache ist. Wir schreiben deshalb unserem Wollen die Rolle zu, als Auslöser für die schließlich bewußt gewordenen Entscheidungen zu fungieren. Diesem Wollen wiederum billigen wir inkonsequenterweise zu, daß es letztinstanzlich und unverursacht, also frei ist.

Sollte diese Interpretation zutreffen, dann wäre unsere Erfahrung, frei zu sein, eine Illusion, die sich aus zwei Quellen nährt: 1.) Der durch die Trennung von bewußten und unbewußten Hirnprozessen widerspruchsfreien Empfindung, alle relevanten Entscheidungsvariablen bewußt gegeneinander abwägen zu können und 2.) der Zuschreibung von Freiheit und Verantwortung durch andere Menschen.'' (S. 49-50)
\end{quote}

\hypertarget{def-ev3}{%
\section{\#3 Der entlarvte Ruck}\label{def-ev3}}

Auszug aus dem Aufsatz „Der entlarvte Ruck. Was sagt Kant den Gehirnforschern‟ von Otfried Höffe \citeyearpar{Höffe2004}

\begin{quote}
``Als Beleg gelten Experimente des Verhaltensphysiologen Benjamin Libet aus dem Jahr 1985 und später der Neuropsychologen Patrick Haggard und Martin Eimer. Eigentlich wollte Libet die Existenz der Willensfreiheit experimentell aufzeigen und untersuchte zu diesem Zweck das willkürliche Auslösen minimaler Bewegungen. Von der autonomen Macht des Geistes überzeugt, erwartete er, daß dem Beginn der Prozesse im Gehirn, dem Aufbau eines sogenannten Bereitschaftspotentials, ein »Willensruck« vorausgehe: Tatsächlich stellte sich das Nichterwartete heraus: Dem Willensruck ging schon ein elektrisches Potential, das Bereitschaftspotential, voraus. Nun fügt sich dieses Ergebnis nahtlos in eine Fülle weiterer neurologischer Erkenntnisse ein: Vor dem Beginn von Handlungen laufen im Gehirn Prozesse ab, die vom Handelnden dann als »willentlich selbst verursacht« berichtet werden. Infolgedessen - behaupten Hirnforscher wie Gerhard Roth und Wolf Singer - habe der Mensch weder Freiheit noch Verantwortung, und der für das Strafrecht wesentliche Begriff der Schuldfähigkeit verliere sein Recht.'' (S. 177)
\end{quote}

\hypertarget{def-ev4}{%
\section{\#4 Gründe zählen}\label{def-ev4}}

Auszüge aus dem Aufsatz ``Gründe zählen. Über einige Schwierigkeiten des Bionaturalismus'' von Lutz Wingert \citeyearpar{Lutz2004}

\begin{quote}
``Folgt man Naturwissenschaftlern wie Gerhard Roth, Wolf Singer oder Wolfgang Prinz, dann hat Sartre unrecht. »Wir sind determiniert«, so Gerhard Roth. Unser Handeln und Urteilen ist fixiert durch organische Prozesse, über deren Wirkung wir als bewußt Urteilende und Handelnde keine Kontrolle haben.'' (S. 195)
\end{quote}

\begin{quote}
``Ein Argument baut auf den oft zitierten Experimenten von Benjamin Libet, Patrick Haggard und Martin Eimer auf. Nach diesen Experimenten lag der bewußte, spontane Entschluß der Testpersonen später als der Zeitpunkt, ab dem neuronale Prozesse zur Ausführung eines willkürmotorischen Ablaufs wie dem des gezielten Tastendrückens einsetzten. Also kann die Handlung des Tastendrückens nicht frei gewesen sein. Denn Handlungen - das ist die obere Prämisse - sind Körperbewegungen, die von bewußten, spontanen Entschlüssen verursacht werden.'' (S. 196-197)
\end{quote}

\hypertarget{def-ev5}{%
\section{\#5 Warum noch debattieren?}\label{def-ev5}}

Auszüge aus dem Aufsatz ``Warum noch debattieren? Determinismus als Diskurskiller'' von Gerhard Kaiser \citeyearpar{Kaiser2004}

\begin{quote}
``Aber derselbe Wolf Singer, der von Komplementarität spricht, verwendet fortgesetzt grenzüberschreitend das Kausalitätsschema (S. 29), nennt Gedanken »Folge neuronaler Prozesse« (S. 15) und weitet die deterministische Position sogar dahin aus, daß »auch die kulturelle Umwelt determiniert« (S. 23).'' (S. 263)
\end{quote}

\begin{quote}
``Willensfreiheit ist relativ - menschliche Freiheit ist nie so gedacht worden, daß dem Menschen alle je denkbaren Optionen auch offenstehen. Sie findet statt innerhalb eines individuellen und historischen Ermöglichungs- und Bedingungsrahmens. Und sie ist, jedenfalls auf der Stufe des reflexiven Bewußtseins, ein Spezifikum des Menschen.'' (S. 265)
\end{quote}

\hypertarget{def-ev6}{%
\section{\#6 Infektion des Geistes}\label{def-ev6}}

Auszüge aus dem Aufsatz ``Infektion des Geistes. Über philosophische Kategorienfehler'' von Gerd Kempermann \citeyearpar{Kempermann2004}

\begin{quote}
``Der freie Wille gehört wie die Menschenwürde in die Karegorie der Konstrukte, die Zuschreibungen sind. Freier Wille bleibt schon in der Eigenwahrnehmung unscharf. Man mag prinzipiell viel auf die eigene Autonomie halten, um sie doch in schwierigen Situationen, nicht nur in Strafverfahren, bereitwillig über Bord zu werfen und auf biologische Entlastungsgründe zurückzugreifen. Wir sind uns unserer Natur und der Abhängigkeit von ihr durchaus bewußt. Aber das befreit uns nicht davon, verantwortlich zu handeln. Denn es ist ja nicht nur Natur in uns.'' (S. 236)
\end{quote}

\begin{quote}
``Wenn man in Zusammenhängen, die wir bisher der Psyche zugeschrieben haben, die Erkenntnisse der Biologie zuläßt, brechen nicht gleich die moralischen Fundamente des Abendlandes weg. Durch Berücksichtigung der »Natur des Menschen« gelingt es oft erst, zu wirklich menschlichen Lösungen zu gelangen. Die forensische Psychiatrie beschäftigt sich seit langem mit diesen Fragen, und auch sie lernt nicht nur durch soziopsychologische Forschung, sondern auch durch Neurobiologie dazu.'' (S. 237)
\end{quote}

\begin{quote}
``Die bestimmende Interaktion zwischen Natur und persönlicher Lebensgeschichte gilt selbstverständlich nicht nur für die Psychiatrie. Bei Herzinfarkten, Typ-II-Diabetes und bestimmten Krebsformen ist das Zusammenspiel von Disposition und Lebensführung Allgemeingut geworden. Was ist Lebensführung anderes als der persönliche, »freie« Umgang mit Disposition, das verantwortliche Leben der eigenen Natur?'' (S. 238)
\end{quote}

\begin{quote}
``Eine Ethik können nur Menschen haben. Biologische Fakten anzuerkennen heißt nicht, der Verantwortung ledig zu sein, eine solche Ethik zu entwickeln.'' (S. 239)
\end{quote}

\hypertarget{def-ev7}{%
\section{\#7 Ick bün all da}\label{def-ev7}}

Auszüge aus dem Aufsatz ``Ick bün all da. Ein neuronales Erregungsmuster'' von Reinhard Brandt \citeyearpar{Brandt2004}

\begin{quote}
``»Wie sünd all da!« Mit ihren raffinierten Horchgeräten und Kernspintomographen hören die Forscher das homerische Gelächter der Zellen über den verwirrten Geist und den Willen der Menschen, die sich frei und selbständig dünken und doch nur ausführen, was im grauen Netzwerk der Zellen zuvor festgelegt wurde. Der Geist gleicht der Fliege, die auf einem Wagenrad sitzt und sich einbildet, das Rad zu bewegen.'' (S. 171)
\end{quote}

\begin{quote}
``Und das Gefühl der Freiheit, das unweigerlich unsere Entschlüsse und die ihnen folgenden Handlungen begleitet, auch dieses unvermeidliche Gefühl ist wohl nichts anderes als eine Illusion, die aus der Retrospektive auch meistens verschwindet: Man hätte damals eben doch nicht anders handeln können. Das Freiheitsgefühl scheint nur der blinde Fleck im Auge zu sein, der uns die determinierenden Ursachen nicht sehen läßt. Im übrigen verhalten wir uns grundsätzlich so, daß wir nicht an die Freiheit der Menschen glauben; wenn das freie Wahlvolk anders stimmt oder der freie Nachbar anders handelt, als erwartet, so fragen wir nach den Ursachen und Motiven wie beim Ausfall von Gas und Strom oder beim auffälligen Verhalten von Hunden und Katzen. Du hättest anders handeln können, das sagen wir bevorzugt, wenn es um Schuldzuweisungen geht, aber sonst sind wir »Deterministen«: Von dem und dem war nichts anderes zu erwarten; Sokrates lügt eben nicht, der kann nicht anders.'' (S. 171-172)
\end{quote}

\begin{quote}
``Jede mitteilbare wahre oder falsche Erkenntnis setzt sich aus Urteilen zusammen wie z. B. »Alles Wissen, über das ein Gehirn verfügt, residiert in seiner funktionellen Architektur«; ein Urteil ist im einfachsten Fall die Einheitsverknüpfung von Subjekt und Prädikat, wobei diese Verknüpfung notwendig entweder bejahend oder verneinend ist (hier also: » Nicht alles Wissen, {[}\ldots{]}« oder »{[}\ldots{]} residiert nicht in seiner Architektur«). Die Verneinung vereint Subjekt und Prädikat im Urteil und behauptet zugleich deren Trennung. Die gegen die Vorstellung vom Primat der Materie oder des Gehirns gegenüber dem Geist gerichtete These lautet: In keiner Gehirnzelle und in keiner Synapse hat man und wird man das Äquivalent eines Urteils, besonders keine Verneinung entdecken. Wer je im Gehirn eine Verneinung auffindet, dem verpfände ich alle Synapsen, die an dieser Zeile beteiligt sind (beim Milliardenaufkommen wird das ja zu verkraften sein). Solange eine Urteils- oder Erkenntnisbildung und besonders eine Verneinung nicht entdeckt wurden, läßt sich der Geist nicht auf noch so dynamische und demokratisch vernetzte Prozesse des Gehirns zurückführen. Sie bieten notwendige, aber keine hinreichenden Bedingungen für den Geist, der nein sagen kann und mit seinem ersten Nein ins Dasein sprang.'' (S. 175)
\end{quote}

\hypertarget{def-ev8}{%
\section{\#8 You don't have free will, but don't worry}\label{def-ev8}}

Ein Video von \href{https://sabinehossenfelder.com}{Dr.~Sabine Hossenfelder} (theoretische Physikerin), in dem sie erklärt warum die Idee des freien Willen mit den derzeit bekannten Naturgesetzen unvereinbar ist \citep{Hossenfelder2020}.

\href{https://www.youtube.com/watch?v=zpU_e3jh_FY}{Video: ``You don't have free will, but don't worry.''}

\hypertarget{def-ev9}{%
\section{\#9 Is Free Will an Illusion?}\label{def-ev9}}

Ein TEDx Talk von \textbf{Nick Jankovic}, in dem er kritisch die Idee diskutiert, dass der freie Wille nichts weiter als eine Illusion ist \citep{Jankovic2021}.

\href{https://www.youtube.com/watch?v=J21bSGGvnD4}{Video: ``Is Free Will an Illusion?''}

\hypertarget{def-ev10}{%
\section{\#10 Freier Wille: Was sagt die Wissenschaft?}\label{def-ev10}}

Ein Video von \textbf{Ralph Caspers}, in dem er versucht den Kenntnisstand der Wissenschaft bzgl. der Frage nach der Existenz des freien Willens zusammenzufassen \citep{Caspers2021}.

\href{https://www.youtube.com/watch?v=45Iut50Cm_Q}{Video: ``Freier Wille: Was sagt die Wissenschaft?''}

\hypertarget{def-expert}{%
\section{Sachverständiger der Verteidigung}\label{def-expert}}

In der Rolle des Sachverständigen \textbf{Benjamin Libet} vertritt \texttt{L.\ G.} die folgende Position:

\begin{quote}
\emph{Die eigene \textbf{Willensfreiheit} ist lediglich eine \textbf{Illusion}, da jegliche Entscheidungsprozesse unbewusst eingeleitet werden.}
\end{quote}

Nutze bitte die folgenden Materialien zur Vorbereitung deiner Aussage.

\hypertarget{def-expert-ev1}{%
\subsection{\#1 Haben wir einen freien Willen?}\label{def-expert-ev1}}

Auszug aus dem Aufsatz ``Haben wir einen freien Willen?'' von Benjamin Libet \citeyearpar{Libet2004}

\begin{quote}
``Ich habe mich dieser Frage auf experimentelle Weise genähert. Freien Willenshandlungen geht eine spezifische elektrische Veränderung im Gehirn voraus (das ›Bereitschaftspotentiak‹, BP), das 550 ms vor der Handlung einsetzt. Menschliche Versuchspersonen wurden sich der Handlungsintention 350-400 ms nach Beginn von BP bewußt, aber 200 ms vor der motorischen Handlung.'' (S.268)
\end{quote}

\hypertarget{def-expert-ev2}{%
\subsection{\#2 Der entlarvte Ruck}\label{def-expert-ev2}}

Auszüge aus dem Aufsatz „Der entlarvte Ruck. Was sagt Kant den Gehirnforschern‟ von Otfried Höffe \citeyearpar{Höffe2004}

\begin{quote}
``Als Beleg gelten Experimente des Verhaltensphysiologen Benjamin Libet aus dem Jahr 1985 und später der Neuropsychologen Patrick Haggard und Martin Eimer. Eigentlich wollte Libet die Existenz der Willensfreiheit experimentell aufzeigen und untersuchte zu diesem Zweck das willkürliche Auslösen minimaler Bewegungen. Von der autonomen Macht des Geistes überzeugt, erwartete er, daß dem Beginn der Prozesse im Gehirn, dem Aufbau eines sogenannten Bereitschaftspotentials, ein »Willensruck« vorausgehe: Tatsächlich stellte sich das Nichterwartete heraus: Dem Willensruck ging schon ein elektrisches Potential, das Bereitschaftspotential, voraus. Nun fügt sich dieses Ergebnis nahtlos in eine Fülle weiterer neurologischer Erkenntnisse ein: Vor dem Beginn von Handlungen laufen im Gehirn Prozesse ab, die vom Handelnden dann als »willentlich selbst verursacht« berichtet werden. Infolgedessen - behaupten Hirnforscher wie Gerhard Roth und Wolf Singer - habe der Mensch weder Freiheit noch Verantwortung, und der für das Strafrecht wesentliche Begriff der Schuldfähigkeit verliere sein Recht.'' (S. 177)
\end{quote}

\hypertarget{def-expert-ev3}{%
\subsection{\#3 Das Experiment von Libet et al.~}\label{def-expert-ev3}}

Auszug aus einem neurowissenschaftlichen Artikel von Brass, Furstenberg und Mele \citeyearpar{Brass2019}.

\begin{quote}
``The experiment by Libet and colleagues (Libet, 1985; Libet et al., 1983) is one of the most iconic experiments of experimental psychology. In this experiment, Libet combined two approaches to study cognitive function, creating an experimental setup that inspired generations of psychologists, neuroscientists and philosophers. The two elements are electrophysiological measurement of brain activity preceding intentional action and subjective timing of conscious intentions. In the experiment, participants have to carry out a simple action such as flexing their wrist or pressing a key. They can choose when to execute this action. While forming the intention to act, participants observe a revolving spot on a clock face. They are instructed to remember the location of the revolving spot when they first experience the decision or urge to act. After they execute the action, they have to indicate on a clock face when exactly they became aware of the intention to act (defined as the''time of conscious intention to act'' in the title of Libet et al., 1983), the so called will judgement or W. Libet'sfirst observation was that W precedes the actual motor response (M) by about 200 ms. This indicates that participants can distinguish the intention to act from the act itself. More important, during the experiment brain activity was recorded with electroencephalography (EEG). Previous research had indicated that voluntary action is preceded by a specific brain wave, the so called `Bereitschaftspotential' or readiness potential (RP, Kornhuber and Deecke, 1964). Libet's ingenious idea was to relate the onset of the RP to W, finding that the onset of the RP precedes W by a few hundred milliseconds. Interestingly, Libet et al.~(1983) distinguished two types of RPs. First, RPs that arise from the spontaneous decision to act without any preplanning of the action (type II RP). Second, RPs that arise in situations where some preplanning occurred (type I RP). The type I RP has an earlier onset than the type II RP. But even in cases where no preplanning was reported by participants, the onset of the RP preceded W by about 500 ms. Based on these findings, Libet argued that the fact that a brain wave precedes the conscious intention by a few hundred milliseconds shows that the conscious intention cannot be the cause of the action but rather is the consequence of brain processes preceding it. In the words of Libet (1985, page: 536): `the brain ``decides'' to initiate or, at least, to prepare to initiate the act before there is any reportable subjective awareness that such a decision has taken place'. Thus, our intuition that we have conscious control over our actions seems to be wrong. Unconscious brain processes apparently determine what we do and we only learn about this immediately before we execute the action.{[}\ldots{]}Libet's basic findings have been replicated numerous times (Haggard and Eimer, 1999; Keller and Heckhausen, 1990; Rigoni et al., 2013b), demonstrating that they are reliable and can be reproduced in different labs.'' (S. 7-9)
\end{quote}

\hypertarget{def-expert-ev4}{%
\subsection{\#4 Freier Wille: Was sagt die Wissenschaft?}\label{def-expert-ev4}}

Ein Video von \textbf{Ralph Caspers}, in dem er versucht den Kenntnisstand der Wissenschaft bzgl. der Frage nach der Existenz des freien Willens zusammenzufassen \citep{Caspers2021}.

\href{https://www.youtube.com/watch?v=45Iut50Cm_Q}{Video: ``Freier Wille: Was sagt die Wissenschaft?''}

\hypertarget{results}{%
\chapter{Ergebnisse der Seminarsitzung}\label{results}}

\hypertarget{pr-ev}{%
\section{Beweisanträge der Staatsanwaltschaft}\label{pr-ev}}

\begin{quote}
``Die Voraussetzungen, die notwendig sind um einen Angeklagten des Mordes schuldig zu sprechen, waren bei der Tat vollständig erfüllt. Zum einen lag der Aspekt der Heimtücke vor. Sigmund F. war zum Zeitpunkt der Tat in einer Besprechung mit Carl-Gustaf J. und dementsprechend wehrlos. Zum anderen zeugte das Tötungsdelikt von besonderer Grausamkeit.''
\end{quote}

\begin{quote}
``Die Quantenphysik zeigt uns das wahrer Zufall existiert. Daher kann es keinen Determinismus geben. Dementsprechend ist davon auszugehen, dass der freie Willen existiert. Die Aussage „Alles ist determiniert'' hat keinen Wahrheitsgehalt, da laut des Determinismus bereits die Tätigung dieser Aussage determiniert gewesen sein muss. Dies ist ein Zirkelschluss.''
\end{quote}

\begin{quote}
``Menschliches Verhalten basiert sowohl auf bewussten als auch auf unbewussten Prozessen. Im Rahmen jener bewussten Prozesse, die zu einem bestimmten Verhalten führen, ist es möglich, sich zwischen richtig und falsch zu entscheiden. Dementsprechend hätte sich der Angeklagte spätestens im Zuge der Ausführung der Tat mittels einer Veto-Entscheidung gegen die Vollendung der Tat entscheiden können.''
\end{quote}

\begin{quote}
``Einen freien Willen gibt es laut Pauen dann, wenn die Handlung autonom ist. Dies ist klar gegeben. Daniel B. hat autonom die Entscheidung getroffen in die Vergangenheit zu reisen und hat die Tat bewusst geplant (Urheberschaft).''
\end{quote}

\begin{quote}
``Daniel B. ist „noch'' kein Mehrfachtöter. Die Gefahr, dass er ohne Strafe ein weiteres Vergehen begehen würde, ist jedoch hoch. Die Wahrscheinlichkeit einer erneuten Straffälligkeit liegt bei ca. 35\%. Besonders Männer sind gefährdet. Deshalb ist eine Gefängnisstrafe unabdingbar. Für die Gesellschaft ist es nicht tragbar, wenn er noch frei rumläuft.''
\end{quote}

\begin{quote}
``Außerdem ist die Grundlage unseres Rechtssystem, dass der Mensch einen freien Willen hat. Dies impliziert, dass der Angeklagte für schuldig gesprochen werden muss. Auch wenn der Angeklagte keinen freien Willen hat, sollte er schuldig gesprochen werden. Es ist notwendig, zumindest die Illusion des freien Willens in unserer Gesellschaft aufrechtzuerhalten.''
\end{quote}

\hypertarget{pr-expert-statement}{%
\subsection{Aussagen des Sachverständigen Immanuel Kant}\label{pr-expert-statement}}

\begin{quote}
\begin{itemize}
\item
  ``Menschen sind autonom, i.e.~selbstbestimmt.''

  \begin{itemize}
  \tightlist
  \item
    ``Maxime, nach der Mensch selbst handelt''
  \end{itemize}
\item
  ``Determinismus ist trotzdem gegeben.''
\item
  ``Willensfreiheit und Determinismus stehen nicht im Widerspruch.''
\end{itemize}
\end{quote}

\hypertarget{pr-expert-cross}{%
\subsection{Kreuzverhör des Sachverständigen Benjamin Libet}\label{pr-expert-cross}}

\begin{quote}
\begin{itemize}
\item
  ``Die Studie weist große methodische Mängel auf.''
\item
  ``Die Instruktion an die Teilnehmenden war: »Schauen Sie auf die Uhr, wenn sie das Gefühl haben, sie möchten gleich die Hand heben.«''

  \begin{itemize}
  \item
    ``Was für ein Gefühl?''
  \item
    ``Wo bzw. wann genau beginnt dieses Gefühl?''
  \item
    ``Geht dem Gefühl ein Wahrnehmungsprozess voraus?''
  \end{itemize}
\item
  ``Es gibt auch andere Studien, die zeigen, dass es zwar keinen freien Willen gibt, aber ein „freies Veto'' zwischen Bereitschaftspotential und Handlung: Free Won't (Mord abbrechen/nicht begehen).''
\end{itemize}
\end{quote}

\hypertarget{def-ev}{%
\section{Beweisanträge der Verteidigung}\label{def-ev}}

\begin{quote}
``\emph{In dubio pro reo} - Im Zweifel für den Angeklagten. Er ist unschuldig bis zum Beweis, dass er einen freien Willen hatte. Man kann freien Willen nicht sehen, anfassen, irgendwie erfahren und daher auch nicht nachweisen. Es gibt keine naturwissenschaftliche, philosophische, etc. stichhaltigen Belege für die Existenz des freien Willens. Wir wollen es nur nicht wahrhaben, dass wir keinen freien Willen haben, weil sich das besser anfühlt, wenn ich jemanden verantwortlich machen kann.''
\end{quote}

\begin{quote}
``Die Idee von Freiheit existiert, aber das heißt ja nicht, dass der freie Wille existiert.''
\end{quote}

\begin{quote}
``Es ist in der Tat so, dass Determinismus existiert. Wir können nur nicht die Zukunft berechnen, weil wir nicht die Mittel dafür besitzen. In der Theorie wäre es jedoch möglich.''
\end{quote}

\begin{quote}
``Der freie Wille ist nur ein Produkt unserer existentiellen Angst. Mit ihm ist es leichter zu leben.''
\end{quote}

\begin{quote}
``Die eigene Willensfreiheit ist nur eine Illusion. Wenn etwas aus Zufall passiert, dann konnte es auch nicht durch den freien Willen beeinflusst werden.''
\end{quote}

\begin{quote}
``Der Angeklagte konnte nicht aus freiem Willen handeln, da es keinen freien Willen gibt. Dementsprechend ist er für sein Handeln nicht verantwortlich und kann daher auch nicht schuldig gesprochen werden.''
\end{quote}

\begin{quote}
``Auch wenn er einen freien Willen haben sollte, ist dieser aufgrund von einer sicherlich vorhandenen psychischen Erkrankung eingeschränkt.''
\end{quote}

\hypertarget{def-expert-statement}{%
\subsection{Aussagen des Sachverständigen Benjamin Libet}\label{def-expert-statement}}

\begin{quote}
\begin{itemize}
\item
  ``Freier Wille wurde experimentell getestet: Entdeckung eines Bereitschaftspotentials vor der Entscheidung eine Handlung auszuführen.''
\item
  ``Daraus resultiert, dass es keinen freien Willen geben kann. Daher ist Daniel B. schuldlos.''
\end{itemize}
\end{quote}

\hypertarget{def-expert-cross}{%
\subsection{Kreuzverhör des Sachverständigen Immanuel Kant}\label{def-expert-cross}}

\begin{quote}
\begin{itemize}
\item
  ``Allein der Glauben an den Freien Willen reicht nicht. Wo sind die empirischen Beweise?''
\item
  ``Nur weil Mensch etwas denken kann, heißt das nicht, dass es das auch gibt.''
\item
  ``Kant kann nicht beweisen, dass Freiheit existiert. Lediglich: Idee der Freiheit''
\end{itemize}
\end{quote}

\hypertarget{pr-plea}{%
\section{Plädoyer der Staatsanwaltschaft}\label{pr-plea}}

\begin{quote}
``Die Annahme der Verteidigung beruht ausschließlich auf dem Argument, dass es keinen freien Willen gibt, da alles determiniert ist.

Sachverständiger Kant hat jedoch betont, dass es auch zufällige Ereignisse gibt. Ein freier Wille ist also nicht ausgeschlossen. Sachverständiger Libet muss bei seinem angeblichen Beweis, dass es keinen freien Willen gibt, einige essentielle Fehler eingestehen. Er ist nicht im Recht, da seine Studie methodische, methologische, logische und theoretische Mängel hat.

In unserer Gesellschaft und vor allem auch in unserem Rechtssystem wird angenommen, dass es einen freien Willen gibt.

Nach gegebener Sachlage, unter Annahme der Existenz des freien Willens laut Gesetzbuch, unter Berücksichtigung der Erfüllung aller Kriterien des Mordes und der erhöhten Gefahr einer erneuten Straffälligkeit, sprechen wir uns deutlich für einen Schuldspruch aus.''
\end{quote}

\hypertarget{def-plea}{%
\section{Plädoyer der Verteidigung}\label{def-plea}}

\begin{quote}
``Naturwissenschaftliche, philosophische und psychologische Begründungen zielen alle auf das gleiche ab: Es gibt keinen freien Willen. In der Physik gilt Determinismus schon als bewiesen. Experimente, wie das Libet-Experiment, muss man nicht kritisieren.

Wir leben nicht in einer Welt, in der man prophylaktisch jemanden wegsperren kann. Die Existenz des freien Willens kann nicht bewiesen werden.

In Zweifel für den Angeklagten.

Daniel B. sollte nicht schuldig gesprochen werden.''
\end{quote}

\hypertarget{sentencing}{%
\section{Urteilsverkündung}\label{sentencing}}

\begin{quote}
``Die Richter*innen sind \textbf{einstimmig für einen Schuldspruch}.

Die Begründungen hierfügen basieren zum einen auf dem Argument, dass es weder Beweise für noch gegen den freien Willen gibt. Ein Schuldspruch ist jedoch für die gesellschaftliche Ordnung notwendig.

Auch wenn Determinismus vorliegt, können wir nicht bewerten, ob Daniel B. nochmal morden wird oder nicht (weil dies nicht berechnet bzw. vorhergesagt werden kann). Ein Schuldspruch ist notwendig, damit Mensch sicherstellen kann, dass Daniel B. keine Gefahr für die Öffentlichkeit darstellt. Daniel B. sollte zuallererst unter psychiatrische Beobachtung gestellt werden. In der Folge sollte ein Gutachten erstellt werden, das eine Aussage darüber trifft, welche Gefahr von Daniel B. ausgeht.''
\end{quote}

  \bibliography{refs.bib}

\end{document}
